% -----------------------------------
% -----------------------------------
% abnTeX2: Normas ABNT NBR 14724:2011 + sugestões FGV/EMAp. 

% Autor: Lauro César Araujo
% Adaptações EMAp: Lucas Machado Moschen 
% Copyright 2012-2018 by abnTeX2 group at http://www.abntex.net.br/ 

%% This work may be distributed and/or modified under the
%% conditions of the LaTeX Project Public License, either version 1.3
%% of this license or (at your option) any later version.
%% The latest version of this license is in
%%   http://www.latex-project.org/lppl.txt
%% and version 1.3 or later is part of all distributions of LaTeX
%% version 2005/12/01 or later.
% ----------------------------------
% ----------------------------------
\documentclass[
	% -- opções da classe memoir --
	12pt,				% tamanho da fonte
	%openright,			% capítulos começam em página ímpar (insere página vazia caso preciso)
	oneside,			% para impressão em recto e verso. Oposto a oneside
	a4paper,			% tamanho do papel. 
	% -- opções da classe abntex2 --
	%chapter=TITLE,		% títulos de capítulos convertidos em letras maiúsculas
	%section=TITLE,		% títulos de seções convertidos em letras maiúsculas
	%subsection=TITLE,	% títulos de subseções convertidos em letras maiúsculas
	%subsubsection=TITLE,% títulos de subsubseções convertidos em letras maiúsculas
	% -- opções do pacote babel --
	english,			% idioma para inglês
	%brazil				% idioma para português
	]{abntex2}

%------------------------------------------------
%-------------- Pacotes necessários -------------
%------------------------------------------------

% Escrita 
\usepackage[T1]{fontenc}
\usepackage[utf8]{inputenc}
\usepackage{lmodern}
\usepackage{microtype} % para melhorias de justificação
\usepackage{indentfirst}
\usepackage{csquotes}

\renewcommand{\ABNTEXchapterfont}{\fontfamily{ptm}\fontseries{b}\selectfont}

% Gráficos 
\usepackage{color}
\usepackage{caption}
\usepackage{subcaption}
\usepackage{multirow}
\usepackage{graphicx}
\usepackage{pdfpages}
\graphicspath{{../../images/}}

% Matemáticos 
\usepackage{bbm}
\usepackage{amsthm, amssymb, amsmath, mathtools}

% Outros 
\usepackage{lipsum}
%\usepackage[textsize=tiny, textwidth=15mm]{todonotes}
\usepackage{multirow}
\usepackage{listings}
\usepackage{lstbayes}

% Citações 
%\usepackage[brazilian,hyperpageref]{backref}
%\usepackage[alf]{abntex2cite}	% Citações padrão ABNT
\usepackage[style=abnt]{biblatex}
\addbibresource{biblio.bib}  

% \renewcommand{\backrefpagesname}{Citado na(s) página(s):~}
% % Texto padrão antes do número das páginas
% \renewcommand{\backref}{}
% % Define os textos da citação
% \renewcommand*{\backrefalt}[4]{
% 	\ifcase #1 %
% 		Nenhuma citação no texto.%
% 	\or
% 		Citado na página #2.%
% 	\else
% 		Citado #1 vezes nas páginas #2.%
% 	\fi}%
% ---

%----------------------------------------
%------- Capa e Folha de Rosto ----------
%----------------------------------------

\newcommand\subtitulo[1]{\def\@subtitulo{#1}}
\newcommand{\imprimirsubtitulo}{\@subtitulo}

\renewcommand{\imprimircapa}{%
	\begin{capa}%
	\center
		\ABNTEXchapterfont\Large \MakeUppercase{\imprimirinstituicao}
		\\\vspace*{4cm}
		{\ABNTEXchapterfont\large \MakeUppercase{\imprimirautor}}
		\vfill
		\begin{center}
		\ABNTEXchapterfont\large\MakeUppercase{\imprimirtitulo}\normalfont\MakeUppercase{:
		\imprimirsubtitulo}
		\end{center}
		\vfill
		\normalfont\large\imprimirlocal
		\\\normalfont\large\imprimirdata
		\vspace*{1cm}
	\end{capa}
}

\makeatletter
\renewcommand{\folhaderostocontent}{
  \begin{center}

    %\vspace*{1cm}
    {\ABNTEXchapterfont\large\MakeUppercase{\imprimirautor}}
	
    \vspace*{\fill}\vspace*{\fill}
    \begin{center}
      \ABNTEXchapterfont\bfseries\large\MakeUppercase{\imprimirtitulo}\normalfont\MakeUppercase{:
      \imprimirsubtitulo}
    \end{center}
    \vspace*{\fill}
	
    \abntex@ifnotempty{\imprimirpreambulo}{%
      \hspace{7.5cm}
      \begin{minipage}{.5\textwidth}
      	\SingleSpacing
         \imprimirpreambulo
         \\\\
         Orientador: \imprimirorientador
       \end{minipage}%
       \vspace*{\fill}
    }%

    % {\large\imprimirorientadorRotulo~\imprimirorientador\par}
    % \abntex@ifnotempty{\imprimircoorientador}{%
    %    {\large\imprimircoorientadorRotulo~\imprimircoorientador}%
    % }%
    \vspace*{\fill}

    {\large\imprimirlocal}
    \par
    {\large\imprimirdata}
    \vspace*{1cm}

  \end{center}
}
\makeatother

\titulo{This is my title}
\autor{This is my name}
\local{Rio de Janeiro}
\data{2021}
\instituicao{%
  Fundação Getulio Vargas \\
  \par
  School of Applied Mathematics
}
\tipotrabalho{Bachelor Dissertation (Undergraduation)}

\preambulo{Bachelor dissertation presented to the School of Applied
Mathematics (FGV/EMAp) to obtain the Bachelor's degree in Applied Mathematics.
\\ \\ Area of Study: what?}

\orientador{Professor}

% Se o seu texto tem subtítulo. 
% Se não tiver, altere o arquivo capa_folha_rosto_tex
\subtitulo{This is my subtitle}

%---------------------------------------------
%-------------------- PDF --------------------
%---------------------------------------------

% alterando o aspecto da cor azul
\definecolor{blue}{RGB}{41,5,195}

% informações do PDF
\makeatletter
\hypersetup{
     	%pagebackref=true,
		pdftitle={\@title}, 
		pdfauthor={\@author},
    	pdfsubject={\imprimirpreambulo},
	    pdfcreator={LaTeX with abnTeX2},
		pdfkeywords={abnt}{latex}{abntex}{abntex2}{trabalho acadêmico}, 
		colorlinks=true,       		% false: boxed links; true: colored links
    	linkcolor=blue,          	% color of internal links
    	citecolor=blue,        		% color of links to bibliography
    	filecolor=magenta,      		% color of file links
		urlcolor=blue,
		bookmarksdepth=4
}
\makeatother

% Posiciona figuras e tabelas no topo da página quando adicionadas sozinhas
% em um página em branco. Ver https://github.com/abntex/abntex2/issues/170
\makeatletter
\setlength{\@fptop}{5pt} % Set distance from top of page to first float
\makeatother

%---------------------------------------
%--------- Mais configurações-----------
%---------------------------------------

% Possibilita criação de Quadros e Lista de quadros.
% Ver https://github.com/abntex/abntex2/issues/176
\newcommand{\quadroname}{Quadro}
\newcommand{\listofquadrosname}{Lista de quadros}

\newfloat[chapter]{quadro}{loq}{\quadroname}
\newlistof{listofquadros}{loq}{\listofquadrosname}
\newlistentry{quadro}{loq}{0}

% configurações para atender às regras da ABNT
\setfloatadjustment{quadro}{\centering}
\counterwithout{quadro}{chapter}
\renewcommand{\cftquadroname}{\quadroname\space} 
\renewcommand*{\cftquadroaftersnum}{\hfill--\hfill}

\setfloatlocations{quadro}{hbtp} % Ver https://github.com/abntex/abntex2/issues/176

%-----------------------------------------------------
%--------------------- Margens -----------------------
%-----------------------------------------------------

\setlrmarginsandblock{3cm}{2cm}{*} % The correct is 3/2
\setulmarginsandblock{3cm}{2cm}{*}
\checkandfixthelayout

%-----------------------------------------------------
%------ Espaçamentos entre linhas e parágrafos -------
%-----------------------------------------------------

% O tamanho do parágrafo é dado por:
\setlength{\parindent}{1.3cm}

% Controle do espaçamento entre um parágrafo e outro:
\setlength{\parskip}{0.2cm}  % tente também \onelineskip

% compila o índice
\makeindex

%------------------------------------------------------
%----------- Personal Definitions ---------------------
%------------------------------------------------------

\newcommand{\R}{\mathbb{R}}
\newcommand{\x}{\boldsymbol{x}}
\newcommand{\N}{\operatorname{Normal}}
\newcommand{\betadist}{\operatorname{Beta}}
\newcommand{\bern}{\operatorname{Bernoulli}}
\newcommand{\tril}{\operatorname{tril}}

\newcommand{\ev}{\mathbb{E}}
\newcommand{\var}{\operatorname{Var}}
\newcommand{\cor}{\operatorname{Cor}}
\newcommand{\cov}{\operatorname{Cov}}

\newtheorem{theorem}{Theorem}[]
\newtheorem{proposition}{Proposition}[]

\theoremstyle{definition}
\newtheorem{definition}{Definition}[section]

\theoremstyle{remark}
\newtheorem*{remark}{Remark}
\newtheorem{assumption}{Assumption}

\newcommand{\improve}[1]{\textcolor{red}{#1}}

\renewcommand{\quadroname}{Chart}

%-------------------------------------------------
%----------------- Document ----------------------
%-------------------------------------------------

\begin{document}

\newcounter{num}
% if num != 1, do not print the pre textual 
\setcounter{num}{1}

\selectlanguage{english}
\frenchspacing 

%----------------------------------------------
%--------------- Pré-textuais -----------------
%----------------------------------------------
%\pretextual

\imprimircapa

\ifnum\value{num}=1
{\imprimirfolhaderosto*

\begin{fichacatalografica}
	\sffamily
	\vspace*{\fill}					% Posição vertical
	\begin{center}					
	\fbox{\begin{minipage}[c][8cm]{13.5cm}		% Largura
	\small
	Ficha catalográfica elaborada pela BMHS/FGV \\

	%\imprimirautor
	Moschen, Lucas Machado % Paginas com as citações na bibl
	
	\hspace{0.5cm} \imprimirtitulo / \imprimirautor. -- \imprimirdata.
	
	\hspace{0.5cm} \thelastpage f.\\
		
	\hspace{0.5cm}
	\parbox[t]{\textwidth}{\imprimirtipotrabalho~--~School of Applied
	Mathematics.}\\
	
	\hspace{0.5cm} Advisor: \imprimirorientador .

	\hspace{0.5cm} Includes bibliography. \\
	
	\hspace{0.5cm}
		1. Bayesian statistics.
		2. Respondent-driven Sampling.
		2. Sensitivity and specificity.
		I. Carvalho, Luiz Max.
		II. School of Applied Mathematics.
		III. \imprimirtitulo 			
	\end{minipage}}
	\end{center}
\end{fichacatalografica}

% Uncomment if you have the pdf 
% \begin{fichacatalografica}
%     \includepdf{fig_ficha_catalografica.pdf}
% \end{fichacatalografica}

%\begin{errata}

\begin{table}[htb]
    \center
    \footnotesize
    \begin{tabular}{|p{1.4cm}|p{1cm}|p{3cm}|p{3cm}|}
    \hline
    \textbf{Folha} & \textbf{Linha} & \textbf{Onde se lê} &
    \textbf{Leia-se}\\
    \hline
    17 & 8 & Matemtica & Matemática \\
    \hline
    \end{tabular}
\end{table}

\end{errata}

\begin{folhadeaprovacao}

    \begin{center}
      {\ABNTEXchapterfont\large\MakeUppercase{\imprimirautor}}
  
      \vspace*{\fill}\vspace*{\fill}
      \begin{center}
        \ABNTEXchapterfont\bfseries\large\MakeUppercase{\imprimirtitulo}\normalfont\MakeUppercase{:
        \imprimirsubtitulo}	
      \end{center}
      \vspace*{\fill}
      
      \hfill
      \begin{minipage}{.7\textwidth}
          \imprimirpreambulo \\ \\
          E aprovado em ?/12/2022 \\
          Pela comissão organizadora
      \end{minipage}%
      \vspace*{\fill}
     \end{center}
  
     \assinatura{\imprimirorientador \\ Escola de Matemática Aplicada} 
     \assinatura{Convidado 1 \\ Instituição 1}
     \assinatura{Convidado 2 \\ Instituição 2}
     %\assinatura{\textbf{Professor} \\ Convidado 3}
     %\assinatura{\textbf{Professor} \\ Convidado 4}
\end{folhadeaprovacao}

% \begin{folhadeaprovacao}
% \includepdf{folhadeaprovacao_final.pdf}
% \end{folhadeaprovacao}

\begin{dedicatoria}
    \vspace*{\fill}
    %\noindent
    \hfill
    \begin{minipage}{.6\textwidth}
     Dedico essa dissertação a todas que lutaram para que eu estivesse aqui. 
    \end{minipage}
\end{dedicatoria}
 
\begin{agradecimentos}
    Lembre de agradecer a quem te apoiou, como, por exemplo, orientador,
    família, agência de fomento, professores conselheiros. 
\end{agradecimentos}

\begin{epigrafe}
\vspace*{\fill}

\begin{flushright}
    \hspace{7.5cm}
    \textit{
        ``If your experiment needs a statistician, you need a better
        experiment.''} \\
        \textit{Ernest Rutherford}
\end{flushright}
\end{epigrafe}

\setlength{\absparsep}{18pt} 
\begin{resumo}[Resumo]
Curvas e superfícies costumam ser visualizados em um espaço ambiente 2D ou 3D.
Esse projeto implementa essa visualização em 3D, e para superfícies,
implementa também o \textit{Geodesic Tracing}: uma visualização intrínseca à
superfície, baseada em curvas geodésicas. Além de curvas e superfícies,
o projeto permite visualizar pontos e vetores. Outros objetos auxiliares podem ser
definidos, como parâmetros(controles deslizantes), funções e grades para
instanciar objetos múltiplas vezes.

Para a especificação dos objetos, uma linguagem textual foi estabelecida,
acompanhada de um compilador capaz de transformar o texto em estruturas de dados
úteis para a renderização. A linguagem é descrita por uma gramática livre-de-contexto
inambígua.

Para a interface gráfica, \textit{OpenGL} é usado para a renderização,
e \textit{Dear ImGUI} é usado para construir os controles e janelas.

O resultado é um sistema de performance em tempo real, testado sem
inconsistências. A estética dos gráficos, da interface e da
linguagem não foram negligenciados, e se tornaram bastante agradáveis.

 Palavras-chave: visualização. curvas. superfícies. compilador.
\end{resumo}

\begin{resumo}[Abstract]
\begin{otherlanguage*}{english}
Curves and surfaces are usually visualized in a 2D or 3D ambient space.
This project implements this visualization in 3D, and for surfaces, also
implements the \textit{Geodesic Tracing}: an intrinsic visualization of the surface,
based on geodesic curves. Points and vectors can also be visualized.
Other auxiliary objects can be defined, like parameters(with slider controls),
functions and grids to instantiate objects.

A textual language was designed for the specification of these objects,
accompanied by a compiler capable of transforming the text into data structures
that are useful for rendering. This language is described by an unambiguous context-free grammar.

For the graphical interface, \textit{OpenGL} is used to do the rendering,
and \textit{Dear ImGUI} is used to build GUI components like buttons and windows.

The result is a system with real-time performance, tested without any inconsistencies.
The aesthetic of the graphics, the interface and the language weren't
overlooked, and became quite pleasing.

 Keywords: visualization. curves. surfaces. compiler.
\end{otherlanguage*}
\end{resumo}

\pdfbookmark[0]{\listfigurename}{lof}
\listoffigures*
\cleardoublepage

% \pdfbookmark[0]{\listofquadrosname}{loq}
% \listofquadros*
% \cleardoublepage

\pdfbookmark[0]{\listtablename}{lot}
\listoftables*
\cleardoublepage

\begin{siglas}
    \item[CDC] Centers for Disease Control and Prevention
    \item[HIV] Human immunodeficiency virus  
  \end{siglas}
  
  \begin{simbolos}
    \item[$\in$] Belongs to 
    \item[$\Sigma_{i=1}^n x_i$] Sum of the variables $x_1, x_2, \dots, x_n$
    \item[$\Pr(A)$] Probability of an event $A$
    \item[$M^T$] Transpose of matrix $M$
    \item[$\ind$] Indicator function 
    \item[$\operatorname{tril}(M)$] Lower triangle matrix of $M$
    \item[$\R_{>0}$] Set of positive real numbers
    \item[$\det(M)$] Determinant of $M$ 
    \item[$\sim$] Is distributed as 
    \item[$iid$] Independent and identically distributed
    \item[$\Phi$] Normal cumulative distribution
    \item[$\exp$] Exponential  
    \item[$\int$] Integral 
    \item[$\ev(X)$] Expected value of random variable $X$ 
    \item[$\var(X)$] Variance of random variable $X$
    \item[$A^*$] Hermitian matrix of $A$ 
  \end{simbolos}


}\fi

\pdfbookmark[0]{\contentsname}{toc}
\tableofcontents*
\cleardoublepage

% ----------------------------------------------------------
% ELEMENTOS TEXTUAIS
% ----------------------------------------------------------
\textual

\chapter{Introdução}
Desenhos de superfícies costumam ser feitos a partir de um ponto de vista do
espaço ambiente 3D ou 2D, como em MatLab, Mathematica e Geogebra.
Uma curva ou superfície é definida numa linguagem e então é renderizada.
Uma forma muito comum de renderização é a discretização da curva ou superfície,
formando segmentos no caso de uma curva, ou triângulos para superfícies.
O usuário então pode interagir com a software, podendo mudar a orientação e a posição
da câmera.

O objetivo desse projeto é a visualização de curvas e superfícies.
Esse projeto também implementa uma visualização de superfícies que não depende de um espaço ambiente.
Para isso, é necessário uma imagem sobre a superfície, para poder observá-la.

A visualização pode ser comparada ao que um ser bidimensional interno à superfície observaria:
simula-se raios de luz partindo da posição do ser, e os pontos iluminados são observados.
Os raios de luz devem seguir caminhos em `linha reta', que minimizam distância.
Para uma superfície qualquer, esses caminhos são chamados de geodésicos,
estudados na geometria diferencial, e descritos no capítulo \ref{geomdiff}.
A visualização, chamada de \textit{geodesic tracing}, renderiza a imagem sobre a superfície,
e suas curvaturas podem ser notadas. Ao se mover, a imagem observada pode se distorcer,
dependendo da curvatura.

A implementação desse projeto é feita em três partes:
compilador, método numérico e interface gráfica.

O compilador fornece uma maneira do usuário definir as superfícies e outros objetos.
O usuário escreve um texto, seguindo algumas regras gramaticais, que então é processado.
A teoria de compiladores é essencial para essa etapa,
principalmente a análise léxica e a análise sintática \cite{Dragon:1}.
O compilador está descrito no capítulo \ref{comp}.
A linguagem, com exemplos de programas, está descrita no capítulo \ref{lang}.

O método numérico se refere à simulação dos raios de luz na superfície.
Um raio de luz é determinado pela posição e direção inicial, que são as condições iniciais.
Um sistema de equações diferenciais ordinárias(equação geodésica \cite{GeomDiff:1})
determina a curva que a luz traça.
Uma solução aproximada da equação é calculada pelo método de Runge-Kutta de ordem 4 \cite{Anal:1}.
O método está descrito no capítulo \ref{numeric}.

A interface gráfica é simples e é construída usando \textit{Dear ImGUI} \cite{ImGui},
uma ferramenta de interface gráfica fácil de usar.
A linguagem de programação escolhida para a implementação desse projeto é \textit{C++},
e para desenhar a interface e os objetos, \textit{OpenGL} é usado.
A interface está descrita no capítulo \ref{interface}.

O objetivo primário desse projeto é a visualização de curvas, superfícies, e o geodesic tracing.
Porém, a estética dos gráficos e da linguagem descritiva, performance
e robustez do sistema também são levados em consideração.

% ----------------------------------------------------------
% Finaliza a parte no bookmark do PDF
% para que se inicie o bookmark na raiz
% e adiciona espaço de parte no Sumário
% ----------------------------------------------------------
\phantompart

\chapter{Desenvolvimento}

Corresponde ao corpo do trabalho, contendo a exposição ordenada e pormemorizada
do assunto. Constam aqui a revisão de literatura, metodologia adotada, os resultados e
sua discussão. Divide-se em seções e subseções. \cite{Robert2007}


\chapter{Conclusions}
\label{ch:conclusions}




% -----------------------------------
% ELEMENTOS PÓS-TEXTUAIS
% -----------------------------------
\postextual
% ----------------------------------

%\bibliography{biblio}
\printbibliography

%\glossary

% ----------------------------------------------------------
% Apêndices
% ----------------------------------------------------------

\begin{apendicesenv}

\partapendices

\input{files/appendix.tex}

\end{apendicesenv}

% ----------------------------------------------------------
% Anexos
% ----------------------------------------------------------

% \begin{anexosenv}

% \partanexos

% \end{anexosenv}

%---------------------------------------------------------------------
% ÍNDICE REMISSIVO
%---------------------------------------------------------------------
\phantompart
\printindex

\end{document}