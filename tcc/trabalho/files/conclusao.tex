\chapter{Conclusão}

A interface gráfica, a linguagem de especificação e o compilador
funcionam sem irregularidades ou inconsistências, e possuem performance em tempo-real.
Além disso, a estética da interface e da linguagem foi levada em consideração:
a linguagem possui \textit{Syntax Highlighting}, e a interface é construída usando uma biblioteca
terceira. O desenho de objetos gráficos é feito com anti-serrilhado, melhorando a estética.
Para o \textit{Geodesic Tracing}, é possível observar os fenômenos de curvatura
correspondentes à teoria da geometria diferencial.

Uma das limitações desse projeto é a linguagem de especificação.
Curvas e superfícies devem ser definidas por equações paramétricas.
Muitas curvas e superfícies são melhores definidas implicitamente.
Outra limitação é o fato do programa ser estrito apenas em forma textual, sem poder utilizar
notações matemáticas como por exemplo, subscritos e expoentes, frações e raízes.

Um possível trabalho futuro é aprimorar a discretização das curvas e superfícies.
Assim, cada segmento ou triângulo se ajusta melhor à curva ou superfície.
Para o desenho das superfícies em 3D, truques de textura podem ser usados.
Com os vetores normais à superfície, é possível fazer um sombreado na superfície(sem projeção de sombras),
aumentando a compreensão e o realismo da forma da superfície.
Com isso, é possível reduzir a quantidade de segmentos ou triângulos da discretização
sem degradar o realismo.
O \textit{Geodesic Tracing} desconsidera o domínio da parametrização da superfície.
Isso significa que superfícies como a faixa de \textit{M\"obius} e a garrafa de \textit{Klein}
não são exibidas de forma totalmente correta, pois deveria ser possível inverter a orientação da câmera
ao dar um passeio paralelo.