\chapter{Linguagem}
\label{lang}

O usuário se comunica com a interface através de um texto, chamado de programa,
que contém os objetos de interesse.
O código \ref{code1} é um exemplo.
\begin{lstlisting}[caption=Exemplo de objetos,label=code1]
#circle and tangents
param r : [/2, 1];
param o : [0, 2pi];
curve c(t) = r(cost, sint, 0), t : [0, 2pi];
grid k : [0, 2pi, 8];
define k2 = k + o;
point p = ck2;
vector v = c'k2 @ p;

#function and surface
#function f(x, y) = x^2+y^2;
#surface s(u,v) = (u,v,f(u,v)), u : [-1, 1], v : [-1, 1];
\end{lstlisting}

A linguagem permite comentários no estilo da linguagem Python, usando \texttt{\#}.

O programa declara os seguintes objetos:

\begin{centering}
\begin{tabularx}{\textwidth}{||c|X||}
    \cline{1-2}
    Objetos & Descrição \\ \hline \hline

    \texttt{r} e \texttt{o} & parâmetros que podem ser alterados na interface.
    Seus valores devem estar nos intervalos indicados. \\ \hline

    \texttt{c} & uma curva parametrizada por \texttt{t}.
    O domínio da parametrização é o intervalo indicado.
    A curva depende do parâmetro \texttt{r}, que foi definido anteriormente. \\ \hline

    \texttt{k} & uma grade de 8 pontos igualmente espaçados no intervalo indicado.
    Uma grade é tratada como uma constante, assim como um parâmetro.
    Se um objeto desenhável depende de uma grade, uma instância é desenhada para cada valor da grade.
    Um objeto pode depender de mais de uma grade. \\ \hline

    \texttt{k2} & uma constante, e não pode ser alterada na interface como os parâmetros.
    Esse tipo de objeto pode ser usado para deixar o programa mais legível. \\ \hline

    \texttt{p} & o ponto da curva \texttt{c} de parâmetro \texttt{t = k2}.
    Esse objeto depende indiretamente de \texttt{k}, então é instanciado 8 vezes. \\ \hline

    \texttt{v} & o vetor tangente da curva \texttt{c} no ponto \texttt{p} 
    e desenhado a partir do mesmo ponto.
    O vetor também depende indiretamente de \texttt{k}, então é desenhado 8 vezes. \\ \hline
    \cline{1-2}
\end{tabularx}
\end{centering}

Os objetos \texttt{f} e \texttt{s} estão comentados, então não são considerados.
Estão presentes apenas para o exemplo ter todos os tipos de objeto.

A linguagem de descrição de objetos não é trivial, nem sua sintaxe matemática,
que possui elementos inventados para esse projeto.
A seguir, uma breve lista de observações:

\begin{itemize}
\item
Os objetos desenháveis são pontos, vetores, curvas e superfícies.
Pontos e vetores podem ser usados em outros objetos, sendo tratados como tuplas.
Por exemplo, \texttt{v} usa o ponto \texttt{p}.
Curvas e superfícies podem ser usadas como funções, mas sem a restrição no domínio.
Por exemplo, \texttt{p} usa \texttt{c} como função.
Um objeto só pode se referir aos objetos definidos anteriormente.

\item
Há duas constantes pré-definidas: \texttt{pi} e \texttt{e}; e diversas funções pré-definidas:
\texttt{sin}, \texttt{cos}, \texttt{tan},
\texttt{exp}, \texttt{log}, \texttt{sqrt} e \texttt{id}.
A função \texttt{id} é a identidade é útil apenas no funcionamento interno do sistema.

\item
Parâmetros e grades podem ser multidimensionais:
\texttt{param T : [0, 1], [0, 1];}. Assim, o objeto \texttt{T} é uma tupla,
e seus elementos podem ser obtidos com \texttt{T\_1} e \texttt{T\_2}.

\item As grades das curvas e superfícies são por padrão 100 e 100x100, respectivamente.
É possível alterar esse valor informando um intervalo do tipo grade: \texttt{[0, 2pi, 250]}.

\item
Há 4 operadores unários. Os operadores \texttt{+} e \texttt{-} são os usuais.
A operação \texttt{*x} representa \texttt{xx}, e \texttt{/x} é igual a \texttt{1/x}.
Para números reais, multiplicação com \texttt{*} e por justaposição são equivalentes.
Porém, para tuplas, \texttt{a*b} representa o produto vetorial
e \texttt{ab} representa o produto escalar.
Assim, \texttt{*x} calcula o quadrado do módulo do vetor \texttt{x}.
Uma função que normaliza vetores pode ser definida assim: \texttt{function N(x) = x/sqrt*x;}.

\item
Numa aplicação de função de uma variável, o argumento não precisa de parênteses:
\texttt{sin x}.
O argumento pode ter operadores unários e até expoentes:
\texttt{sin -x\textasciicircum2 = sin(-x\textasciicircum2)}.
Deve-se tomar cuidado com expoentes: \texttt{sin(x)\textasciicircum y = sin(x\textasciicircum y)}.
Para a exponenciação de uma aplicação, deve-se usar a sintaxe: \texttt{sin\textasciicircum2 x}.

\item
Não é sempre necessário uma separação entre identificadores.
Por exemplo, considere \texttt{sinx}.
Caso haja um termo chamado \texttt{sinx} definido, esse seria o identificador reconhecido.
Caso contrário, \texttt{sin x} será reconhecido,
mesmo que \texttt{sinx} seja definido posteriormente
(\texttt{sinx} seria reconhecido apenas depois de sua definição).
Em geral, o maior identificador definido será reconhecido.
\end{itemize}

\section{Gramática formal}
O programa deve seguir uma gramática formal,
que especifica a sintaxe das declarações dos objetos e das expressões matemáticas.
As expressões matemáticas podem seguir uma notação mais natural
que as de várias linguagens de programação.
Por exemplo, há multiplicação por justaposição: \texttt{3x = 3*x};
e a aplicação de funções de uma variável não exige parênteses: \texttt{sin-x = sin(-x)}.

A gramática livre de contexto é definida pelo código \ref{gram} \cite{Dragon:1}.
Uma parte da sintaxe das expressões matemáticas foi baseada na gramática da linguagem C \cite{CGram}.

\begin{lstlisting}[caption=Gramática livre de contexto,label=gram]
PROG    = DECL PROG | ;

DECL    = "param"     id ":" INTS ";" ;
DECL    = "grid"      id ":" GRIDS ";" ;
DECL    = "define"    id "=" EXPR ";" ;
DECL    = "curve"     FDECL "," TINTS ";" ;
DECL    = "surface"   FDECL "," TINTS ";" ;
DECL    = "function"  FDECL ";" ;
DECL    = "point"     id "=" EXPR ";" ;
DECL    = "vector"    id "=" EXPR "@" EXPR ";" ;

FDECL   = id "(" IDS ")" "=" EXPR ;
IDS     = IDS "," id | id ;
INT     = "[" EXPR "," EXPR "]" ;
GRID    = "[" EXPR "," EXPR "," EXPR "]" ;
TINT    = id ":" INT | id ":" GRID ;
INTS    = INTS "," INT | INT ;
TINTS   = TINTS "," TINT | TINT ;
GRIDS   = GRIDS "," GRID | GRID ;

EXPR    = ADD ;
ADD     = ADD "+" JUX | ADD "-" JUX | JUX ;
JUX     = JUX MULT2 | MULT ;
MULT    = MULT "*" UNARY | MULT "/" UNARY | UNARY ;
MULT2   = MULT2 "*" UNARY | MULT2 "/" UNARY | APP ;
UNARY   = "+" UNARY | "-" UNARY | "*" UNARY | "/" UNARY | APP;
APP     = FUNC UNARY | POW ;
FUNC    = FUNC2 "^" UNARY | FUNC2 ;
FUNC2   = FUNC2 "_" var | FUNC2 "'" | func ;

POW     = COMP "^" UNARY | COMP ;
COMP    = COMP "_" num | FACT ;
FACT    = const | num | var
        | "(" TUPLE ")" | "[" TUPLE "]" | "{" TUPLE "}" ;
TUPLE   = ADD "," TUPLE | ADD ;

\end{lstlisting}

Os termos em maiúsculo(não-terminais) representam variáveis gramaticais.
O lado direito de uma igualdade especifica as possíveis formas sentenciais
que um não-terminal pode assumir,
separadas por uma barra vertical ou em diferentes equações.
Uma forma sentencial(ou produção) é uma sequência de terminais e não terminais, possivelmente vazia.
Por exemplo, \texttt{MULT} possui 3 formas:
\texttt{MULT * UNARY}, \texttt{MULT / UNARY} e \texttt{UNARY}.
Cada forma tem um significado diferente.
Uma forma pode ser vazia, como ocorre para \texttt{PROG}.

Os símbolos entre aspas representam textos literais,
e os termos em minúsculo(terminais) representam uma classe de ``palavras'':
Por exemplo, \texttt{num} representa um número e \texttt{var} o nome de uma variável.

O termo \texttt{PROG} representa um programa completo,
que é uma sequência de declarações(\texttt{DECL}).
O termo \texttt{EXPR} representa uma expressão matemática.
Os símbolos abaixo de \texttt{EXPR} definem a sintaxe das operações,
suas ordens de precedência e associatividades.

Para formar um programa gramaticalmente correto, inicia-se com o símbolo \texttt{PROG}.
Cada não-terminal deve ser substituído no lugar por uma de suas formas sentenciais.
O programa estará completo quando não houver mais não-terminais.

Para extraír o significado de um programa, o processo contrário deve ser feito.
É necessário encontrar uma maneira de se obter o programa a partir de \texttt{PROG}.
Para um programa coeso(gramaticalmente correto), sempre há uma maneira, que é única.

Algumas transformações nessa gramática a torna LL1,
uma propriedade que garante que é possível fazer um \textit{parsing} preditivo.
Uma consequência desse fato é a gramática não ser ambígua.
A verificação foi feita em \cite{GramCheck}.
Em uma iteração anterior da gramática, a potenciação de funções era associativa à esquerda,
enquanto a potenciação de números era à direita.
Isso causou uma ambiguidade que não foi detectada no momento.
Ela só foi descoberta ao tentar verificar a propriedade LL1, que falhou.


A tabela \ref{order} descreve as operações e suas ordens de precedência, com base na gramática.

\begin{table}[ht]
\caption{Ordem das operações}
\label{order}
\begin{centering}
\begin{tabularx}{\textwidth}{||c|c|c|c|X||}
    \cline{1-5}
    Operações & Aridade & Associatividade & Exemplo & Descrição \\ \hline \hline

    \texttt{() [] \{\}} & Unário &  & \texttt{(expr)} & Isola a expressão interna \\ \hline
    , & Binário & Esquerda & \texttt{(a,b,c)} & Adiciona uma elemento à tupla(dentro de parênteses) \\ \hline
    + - & Binário & Esquerda & \texttt{a+b} & Soma e subtração \\ \hline
    \textit{justaposição} & Binário & Esquerda & \texttt{ab} & Multiplicação \\ \hline
    * / & Binário & Esquerda & \texttt{a*b} & Multiplicação e Divisão \\ \hline
    + - * / & Unário &  & \texttt{-x}, \texttt{*v} & Positivo, Negativo, Quadrado e Recíproco \\ \hline
    \textit{aplicação} & Binário & Esquerda & \texttt{sin x} & Aplicação de função \\ \hline
    \textasciicircum & Binário & Direita & \texttt{a\textasciicircum b} & Potenciação \\ \hline
    \_ & Unário & & \texttt{(1, 2, 3)\_2} & Elemento de tupla \\ \hline
    ' \_ & Unário & & \texttt{sin'x + f\_z(3)} & Derivada Total e Parcial \\ \hline
    \cline{1-5}
\end{tabularx}
\end{centering}
\end{table}

\section{Compilador}
O processo de compilação não é trivial, e é dividido em 3 estágios:
\begin{itemize}
    \item análise léxica: reconheçe as ``palavras'' que compõe um programa,
    ignorando comentários e espaços em branco. É capaz de identificar números, 
    constantes, nomes de objetos, e pontuação.

    \item análise sintática: parser reconheçe a estrutura do programa: as declarações dos objetos
    e as expressões matemáticas. A gramática \ref{gram} é usada como base.

    \item análise semântica e síntese: gera todas as estruturas de dados
    necessárias para a visualização dos objetos. Verifica também a semântica do programa,
    detectando erros que não podem ser verificados com noções gramaticais.
\end{itemize}

Os estágios se comunicam entre si, e compartilham uma tabela de símbolos.
A tabela especifica quais identificadores estão definidos, e quais são seus tipos.
Uma declaração de um objeto de nome \texttt{X} insere \texttt{X} na tabela, e o associa ao tipo do objeto.
Para isso, \texttt{X} não pode estar definido antes de ser declarado.
Identificadores não definidos são reconhecidos como \texttt{id} na gramática.
Após a declaração, o mesmo identificador passa a ser reconhecido como \texttt{const} ou \texttt{func},
dependendo do tipo de objeto.
Isso significa que os estágios devem ser executados em conjunto,
pois o nome de uma função é gramaticamente diferente do nome de uma constante, por exemplo.
A tabela é inicializada com as constantes e funções pré-definidas e palavras-chave.

A síntese é responsável por computar as derivadas simbolicamente,
e compilar as expressões matemáticas para a linguagem de \textit{shader} do OpenGL: \textit{GLSL}.
Para os objetos desenháveis, suas fórmulas são usadas para determinar seus desenhos.
Para as superfícies, o geodeic tracing também é compilado.

%%%....
%Implementação de sintaxe que considera a estetica natural da escrita matemática que precisa ser traduzida para sintaxe de linguagem computacional. Isso é feito pelo parser.
%A justificativa para essa abordagem é motivada pela experiencia de especificar desenhos de objetos gráficos em bibliotecas de uso corrente, tais como sagemath, manin etc.
%Isso justifica a especificação de uma gramática livre de contexto. Explicitada a seguir.
%Trabalhos futuros: grade variável
%Sobre o texto que descreve a gramática
%Fazer referencia à calculadora C++
%Fazer referência ao site que valida a não ambiguidade da gramática
%Por ter influenciado na sintaxe proposta (como?)
%Tentar lembrar um exemplo de ambiguidade que foi resolvido para a versão atual da gramática.
%Descrever o LL1 como certificado de não ambiguidade.