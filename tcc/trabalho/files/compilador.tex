\chapter{Compilador}
\label{compiler}

O usuário se comunica com a interface através de um texto, chamado de programa,
que contém os objetos de interesse.
Por exemplo:
\begin{lstlisting}[caption=Exemplo de objetos,label=code1]
#circle and tangents
param r : [/2, 1];
param o : [0, 2pi];
curve c(t) = r(cost, sint, 0), t : [0, 2pi];
grid k : [0, 2pi, 8];
define k2 = k + o;
point p = ck2;
vector v = c'k2 @ p;

#function and surface
#function f(x, y) = x^2+y^2;
#surface s(u,v) = (u,v,f(u,v)), u : [-1, 1], v : [-1, 1];
\end{lstlisting}

A linguagem permite comentários no estilo da linguagem Python, usando \texttt{\#}.

O programa declara os seguintes objetos:

\begin{centering}
\begin{tabularx}{\textwidth}{||c|X||}
    \cline{1-2}
    Objetos & Descrição \\ \hline \hline
    
    \texttt{r} e \texttt{o} & parâmetros que podem ser alterados na interface.
    Seus valores devem estar nos intervalos indicados \\ \hline
    
    \texttt{c} & uma curva parametrizada por \texttt{t}.
    O domínio da parametrização é o intervalo indicado.
    A curva depende do parâmetro \texttt{r}, que foi definido anteriormente. \\ \hline
    
    \texttt{k} & uma grade de 8 pontos igualmente espaçados no intervalo indicado.
    Uma grade é tratada como uma constante, assim como um parâmetro.
    Se um objeto desenhável depende de uma grade, uma instância é desenhada para cada valor da grade.
    Um objeto pode depender de mais de uma grade. \\ \hline
    
    \texttt{k2} & uma constante, e não pode ser alterada na interface como os parâmetros.
    Esse tipo de objeto pode ser usado para deixar o programa mais legível. \\ \hline
    
    \texttt{p} & o ponto da curva \texttt{c} de parâmetro \texttt{t = k2}.
    Esse objeto depende indiretamente de \texttt{k}, então é instanciado 8 vezes. \\ \hline
    
    \texttt{v} & o vetor tangente da curva \texttt{c} no ponto \texttt{p} 
    e desenhado a partir do mesmo ponto.
    O vetor também depende indiretamente de \texttt{k}, então é desenhado 8 vezes. \\ \hline
    \cline{1-2}
\end{tabularx}
\end{centering}

Os objetos \texttt{f} e \texttt{s} estão comentados, então não são considerados.
Estão presentes apenas para o exemplo ter todos os tipos de objeto.

Os objetos desenháveis são pontos, vetores, curvas e superfícies.
Pontos e vetores podem ser usados como valores em outros objetos.
Pontos representam suas posições e vetores seus deslocamentos.
Curvas e superfícies podem ser usadas como funções, sem a restrição no domínio.

Há duas constantes pré-definidas: \texttt{pi} e \texttt{e}; e diversas funções pré-definidas:
\texttt{sin}, \texttt{cos}, \texttt{tan},
\texttt{exp}, \texttt{log}, \texttt{sqrt} e \texttt{id}.
A função \texttt{id} é a identidade é útil apenas no funcionamento interno do sistema.

Parâmetros e grades podem ser multidimensionais:
\texttt{param T : [0, 1], [0, 1];}. Assim, o objeto \texttt{T} é uma tupla,
e seus elementos podem ser obtidos com \texttt{T\_1} e \texttt{T\_2}.

Há 4 operadores unários: \texttt{+} e \texttt{-} são os usuais.
A operação \texttt{*x} representa \texttt{xx}, e \texttt{/x} é igual a \texttt{1/x}.
Para números reais, multiplicação com \texttt{*} e por justaposição são equivalentes.
Porém, para tuplas, \texttt{a*b} representa o produto vetorial
e \texttt{ab} representa o produto escalar.
Assim, \texttt{*x} calcula o quadrado do módulo do vetor \texttt{x}.

\section{Gramática formal}
O programa deve seguir uma gramática formal, 
que especifica a sintaxe das declarações dos objetos e das expressões matemáticas.
As expressões matemáticas podem seguir uma notação mais natural
que as de várias linguagens de programação.
Por exemplo, há multiplicação por justaposição: \texttt{3x = 3*x};
e a aplicação de funções não exige parênteses: \texttt{sin-x = sin(-x)}.

A gramática livre de contexto é definida pelo código \ref{gram} \cite{Dragon:1}.
A sintaxe das expressões matemáticas foi baseada na gramática da linguagem C \cite{CGram}.

\newpage
\begin{lstlisting}[caption=Gramática livre de contexto,label=gram]
PROG    = DECL PROG | ;

DECL    = "param"     id ":" INTS ";" ;
DECL    = "grid"      id ":" GRIDS ";" ;
DECL    = "define"    id "=" EXPR ";" ;
DECL    = "curve"     FDECL "," TINTS ";" ;
DECL    = "surface"   FDECL "," TINTS ";" ;
DECL    = "function"  FDECL ";" ;
DECL    = "point"     id "=" EXPR ";" ;
DECL    = "vector"    id "=" EXPR "@" EXPR ";" ;

FDECL   = id "(" IDS ")" "=" EXPR ;
IDS     = IDS "," id | id ;
INT     = "[" EXPR "," EXPR "]" ;
GRID    = "[" EXPR "," EXPR "," EXPR "]" ;
TINT    = id ":" INT | id ":" GRID ;
INTS    = INTS "," INT | INT ;
TINTS   = TINTS "," TINT | TINT ;
GRIDS   = GRIDS "," GRID | GRID ;

EXPR    = ADD ;
ADD     = ADD "+" JUX | ADD "-" JUX | JUX ;
JUX     = JUX MULT2 | MULT ;
MULT    = MULT "*" UNARY | MULT "/" UNARY | UNARY ;
MULT2   = MULT2 "*" UNARY | MULT2 "/" UNARY | APP ;
UNARY   = "+" UNARY | "-" UNARY | "*" UNARY | "/" UNARY | APP;
APP     = FUNC UNARY | POW ;
FUNC    = FUNC2 "^" UNARY | FUNC2 ;
FUNC2   = FUNC2 "_" var | FUNC2 "'" | func ;

POW     = COMP "^" UNARY | COMP ;
COMP    = COMP "_" num | FACT ;
FACT    = const | num | var
        | "(" TUPLE ")" | "[" TUPLE "]" | "{" TUPLE "}" ;
TUPLE   = ADD "," TUPLE | ADD ;
    
\end{lstlisting}

Os termos em maiúsculo(não-terminais) representam variáveis gramaticais.
O lado direito de uma igualdade especifica as possíveis formas sentenciais
que um não-terminal pode assumir, 
separadas por uma barra vertical ou em diferentes equações.
Por exemplo, \texttt{MULT} possui 3 formas:
\texttt{MULT * UNARY}, \texttt{MULT / UNARY} e \texttt{UNARY}.
Cada forma tem um significado diferente.
Uma forma pode ser vazia, como ocorre para \texttt{PROG}.

Os símbolos entre aspas representam textos literais,
e os termos em minúsculo(terminais) representam uma classe de ``palavras'':
Por exemplo, \texttt{num} representa um número e \texttt{var} um nome de uma variável.

O termo \texttt{PROG} representa um programa completo,
que é uma sequência de declarações(\texttt{DECL}).
O termo \texttt{EXPR} representa uma expressão matemática.
Os símbolos abaixo de \texttt{EXPR} definem a sintaxe das operações,
suas ordens de precedência e associatividades.

Um programa coeso é formado a partir de \texttt{PROG}.
Enquando houver não-terminais, deve-se substituí-los por uma de suas formas.

Para extraír o significado de um programa, o processo contrário deve ser feito. 
É necessário encontrar uma maneira de se obter o programa a partir de \texttt{PROG}.
Para um programa coeso, sempre há uma maneira e essa é única.
Algumas transformações nessa gramática a torna LL1,
uma propriedade que garante que o \textit{parsing} pode ser feito de forma fácil e rápida.
Além disso, LL1 garante que a gramática não é ambígua. \cite{GramCheck}

A tabela \ref{order} descreve as operações e suas ordens de precedência, com base na gramática.

\begin{table}[ht]
\caption{Ordem das operações}
\label{order}
\begin{centering}
\begin{tabularx}{\textwidth}{||c|c|c|c|X||}
    \cline{1-5}
    Operações & Aridade & Associatividade & Exemplo & Descrição \\ \hline \hline

    \texttt{() [] \{\}} & Unário &  & \texttt{(expr)} & Isola a expressão interna \\ \hline
    , & Binário & Esquerda & \texttt{(a,b,c)} & Adiciona uma elemento à tupla(dentro de parênteses) \\ \hline
    + - & Binário & Esquerda & \texttt{a+b} & Soma e subtração usuais \\ \hline
    \textit{justaposição} & Binário & Esquerda & \texttt{ab} & Multiplicação \\ \hline
    * / & Binário & Esquerda & \texttt{a*b} & Multiplicação e Divisão \\ \hline
    + - * / & Unário &  & \texttt{-x}, \texttt{*v} & Positivo, Negativo, Quadrado e Recíproco \\ \hline
    \textit{aplicação} & Binário & Esquerda & \texttt{sin x} & Soma e subtração usuais \\ \hline
    \textasciicircum & Binário & Direita & \texttt{a\textasciicircum b} & Potenciação \\ \hline
    \_ & Unário & & \texttt{(1, 2, 3)\_2} & Elemento da tupla \\ \hline
    ' \_ & Unário & & \texttt{sin'x + f\_z(3)} & Derivada Total e Parcial \\ \hline
    \cline{1-5}
\end{tabularx}
\end{centering}
\end{table}


%%%....
%Implementação de sintaxe que considera a estetica natural da escrita matemática que precisa ser traduzida para sintaxe de linguagem computacional. Isso é feito pelo parser.
%A justificativa para essa abordagem é motivada pela experiencia de especificar desenhos de objetos gráficos em bibliotecas de uso corrente, tais como sagemath, manin etc.
%Isso justifica a especificação de uma gramática livre de contexto. Explicitada a seguir.
%Trabalhos futuros: grade variável
%Sobre o texto que descreve a gramática
%Fazer referencia à calculadora C++
%Fazer referência ao site que valida a não ambiguidade da gramática
%Por ter influenciado na sintaxe proposta (como?)
%Tentar lembrar um exemplo de ambiguidade que foi resolvido para a versão atual da gramática.
%Descrever o LL1 como certificado de não ambiguidade.