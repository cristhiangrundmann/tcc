\setlength{\absparsep}{18pt} 
\begin{resumo}[Resumo]
Curvas e superfícies costumam ser visualizados em um espaço ambiente 2D ou 3D.
Esse projeto implementa essa visualização em 3D, e para superfícies,
implementa também o \textit{Geodesic Tracing}: uma visualização intrínseca à
superfície, baseada em curvas geodésicas. Além de curvas e superfícies,
o projeto permite visualizar pontos e vetores. Outros objetos auxiliares podem ser
definidos, como parâmetros(controles deslizantes), funções e grades para
instanciar objetos múltiplas vezes.

Para a especificação dos objetos, uma linguagem textual foi estabelecida,
acompanhada de um compilador capaz de transformar o texto em estruturas de dados
úteis para a renderização. A linguagem é descrita por uma gramática livre-de-contexto
inambígua.

Para a interface gráfica, \textit{OpenGL} é usado para a renderização,
e \textit{Dear ImGUI} é usado para construir os controles e janelas.

O resultado é um sistema de performance em tempo real, testado sem
inconsistências. A estética dos gráficos, da interface e da
linguagem não foram negligenciados, e se tornaram bastante agradáveis.

 Palavras-chave: visualização. curvas. superfícies. compilador.
\end{resumo}

\begin{resumo}[Abstract]
\begin{otherlanguage*}{english}
Curves and surfaces are usually visualized in a 2D or 3D ambient space.
This project implements this visualization in 3D, and for surfaces, also
implements the \textit{Geodesic Tracing}: an intrinsic visualization of the surface,
based on geodesic curves. Points and vectors can also be visualized.
Other auxiliary objects can be defined, like parameters(with slider controls),
functions and grids to instantiate objects.

A textual language was designed for the specification of these objects,
accompanied by a compiler capable of transforming the text into data structures
that are useful for rendering. This language is described by an unambiguous context-free grammar.

For the graphical interface, \textit{OpenGL} is used to do the rendering,
and \textit{Dear ImGUI} is used to build GUI components like buttons and windows.

The result is a system with real-time performance, tested without any inconsistencies.
The aesthetic of the graphics, the interface and the language weren't
overlooked, and became quite pleasing.

 Keywords: visualization. curves. surfaces. compiler.
\end{otherlanguage*}
\end{resumo}