\setlength{\absparsep}{18pt} 
\begin{resumo}[Resumo]
Curvas e superfícies costumam ser visualizados em um espaço ambiente 2D ou 3D.
Esse projeto implementa essa visualização em 3D, e para superfícies,
implementa também o \textit{Geodesic Tracing}: uma visualização intrínseca à
superfície, baseada em curvas geodésicas. Além de curvas e superfícies,
o projeto permite visualizar pontos e vetores. Outros objetos auxiliares podem ser
definidos, como parâmetros(controles deslizantes), funções e grades para
instanciar objetos múltiplas vezes.

Para a especificação dos objetos, uma linguagem textual foi estabelecida,
acompanhada de um compilador capaz de tranformar o texto em estruturas de dados
úteis para a renderização.

Para a interface gráfica, \textit{OpenGL} é usado para a renderização,
e \textit{Dear ImGUI} é usado para construir os controles e janelas.

...

 Palavras-chave: visualização. curvas. superfícies. compilador.
\end{resumo}

\begin{resumo}[Abstract]
 \begin{otherlanguage*}{english}
  (ENGLISH)
 \end{otherlanguage*}

 Keywords: ...
\end{resumo}