\chapter{Curvas e Superfícies}
\label{geomdiff}
O objetivo primário do projeto é visualizar curvas e superfícies.
As curvas são visualizadas apenas em espaço 3D.
As superfícies são visualizadas em espaço 3D e em \textit{geodesic tracing}.

\section{Curvas}
Há duas principais maneiras de se definir uma curva na geometria analítica:
por parametrização e por equação. Esse trabalho apenas consideras curvas paramétricas.

Uma curva pode ser parametrizada por um número real.
Formalmente, uma parametrização é uma função $\gamma : I \rightarrow \mathbb{R}^n$, onde $I$ é um intervalo real.
Nesse trabalho, o intervalo é fechado, e $n=3$.

As curvas são desenhadas através de vários segmentos.
Dada uma partição de $I$ de $k$ pontos,
pode-se aproximar a curva pelos segmentos de extremidade $\gamma(t_{i})$ e $\gamma(t_{i-1})$ para $i<k$,
onde $t_i$ é o i-ésimo ponto da partição. Nesse trabalho, a partição depende apenas de $k$ e é uniforme.

O vetor tangente pode ser calculado com $\gamma'(t)$.

\section{Superfícies}
Assim como as curvas, apenas superfícies parametrizadas serão consideradas nesse trabalho:
$\sigma :  I_1 \times I_2 \rightarrow \mathbb{R}^3$, onde $I_1$ e $I_2$ são intervalos fechados reais.

As superfícies são desenhadas através de vários triângulos,
a partir de partições dos intervalos $I_1$ e $I_2$.
Juntas, as partições formam uma grade de retângulos, e cada retângulo pode ser dividido em 2 triângulos.
Esses são os triângulos desenhados.

Os vetores tangentes nas direções coordenadas são as derivadas parciais $\sigma_u(u,v)$ e $\sigma_v(u,v)$, onde os parâmetros são $u$ e $v$.
Nesse projeto, os parâmetros podem ter nomes quaisquer.

\subsection{Primeira forma fundamental}
Supondo que a superfície seja diferenciável e com vetores tangentes linearmente independentes,
a primeira forma fundamental no ponto paramétrico $(u,v)$ é definida como
\[
    \left[
        \begin{array}{cc}
            \sigma_u \cdot \sigma_u & \sigma_u \cdot \sigma_v \\
            \sigma_v \cdot \sigma_u & \sigma_v \cdot \sigma_v
        \end{array}
    \right]
    = 
    \left[
        \begin{array}{cc}
            E & F \\
            F & G
        \end{array}
    \right]
\]
onde as funções são todas aplicadas no ponto $(u,v)$.

Os vetores $\sigma_u$ e $\sigma_v$ formam uma base do espaço tangente.
O produto escalar de dois vetores tangentes 
$x = x_1 \sigma_u + x_2 \sigma_v$ e $y = y_1 \sigma_u + y_2 \sigma_v$ pode ser calculado da seguinte forma:
\[x\cdot y = (x_1 \sigma_u + x_2 \sigma_v) \cdot (y_1 \sigma_u + y_2 \sigma_v)\]
\[x\cdot y = x_1 y_1 E + x_1 y_2 F + x_2 y_1 F + x_2 y_2 G\]

O produto depende apenas dos coeficientes e da primeira forma fundamental.
Isso significa que distâncias e ângulos podem ser calculados
sem se referir ao espaço ambiente da parametrização, ou seja, de forma intrínseca.

\subsection{Equação geodésica}

\subsection{Solução Numérica}

\subsection{Geodesic Tracing}
