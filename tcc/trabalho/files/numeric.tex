\chapter{Cálculo numérico de geodésicas}
\label{numeric}

A equação \ref{geoeq2} caracteriza as curvas geodésicas que devem ser 
computadas para a visualização do Geodesic Tracing.
Nem sempre é possível resolver a equação de forma analítica, então uma aproximação
deve ser computada no lugar.

O método numérico escolhido é o método de Runge-Kutta de ordem 4 \cite{Anal:1}.
Seja $y' = f(t, y)$ uma equação diferencial para $y(t)$ com valores iniciais $t_0$ e $y_0$.
O método consiste em aproximar $y_1 = y(t_0+h)$, para um passo $h>0$.
A aproximação é feita pelas equações \ref{runge}.

\begin{equation}
\label{runge}
\begin{split}
k_1 & = hf(t_0, y_0) \\
k_2 & = hf(t_0 + h/2, y_0 + k_1/2) \\
k_3 & = hf(t_0 + h/2, y_0 + k_2/2) \\
k_4 & = hf(t_0 + h, y_0 + k_3) \\
y_1 & = y_0 + (k_1 + 2k_2 + 2k_3 + k_4)/6 \\
t_1 & = t_0 + h
\end{split}
\end{equation}

As equações devem ser adaptadas para o sistema \ref{geoeq2}, pois é de 
segunda ordem e possui mais de uma equação.
O sistema pode ser escrito vetorialmente:
\[y' = (u, v, u', v')' = f(u, v, u', v') = f(y)\]

Ou ainda 
\[y' = (\text{pos}, \text{vel})' = f(\text{pos}, \text{vel}) = f(y)\]

A função $f$ computa $(u', v', u'', v'') = (\text{vel}, \text{acc})$, onde $\text{vel}$ é
igual ao argumento $\text{vel}$ de $f$. Os valores $u''$ e $v''$ são calculados conforme a equação
\ref{geoeq2}, usando $\text{pos}$ e $\text{vel}$.
Note que a função $f$ não depende do parâmetro $t$ da curva.

O sistema pode ser resolvido numericamente por Runge-Kutta, ignorando a variável $t$,
e fazendo $y$ como $(\text{pos}, \text{vel})$.

\begin{equation}
\label{method}
\begin{split}
    p_1 & = h\text{vel}_0 \\
    v_1 & = hg(\text{pos}_0, \text{vel}_0) \\
    p_2 & = h(\text{vel}_0+v_1/2) \\
    v_2 & = hg(\text{pos}_0+p_1/2, \text{vel}_0+v_1/2) \\
    p_3 & = h(\text{vel}_0+v_2/2) \\
    v_3 & = hg(\text{pos}_0+p_2/2, \text{vel}_0+v_2/2) \\
    p_4 & = h(\text{vel}_0+v_3) \\
    v_4 & = hg(\text{pos}_0+p_3, \text{vel}_0+v_3) \\
    \text{pos}_1 & = (p_1+2p_2+2p_3+p_4)/6 \\
    \text{vel}_1 & = (v_1+2v_2+2v_3+v_4)/6
\end{split}
\end{equation}