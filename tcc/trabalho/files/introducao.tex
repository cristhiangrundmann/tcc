\chapter{Introdução}
Desenhos de superfícies costumam ser feitos a partir de um ponto de vista do
espaço ambiente 3D.
Esse projeto implementa uma visualização de superfícies que não depende de um espaço ambiente.

A visualização pode ser comparada ao que um ser bidimensional interno à superfície observaria:
simula-se raios de luz partindo da posição do ser, e os pontos iluminados são observados.
Os raios de luz devem seguir caminhos em `linha reta', que minimizam distância.
Para uma superfície qualquer, esses caminhos são chamados de geodésicos,
e são estudados na geometria diferencial.
A visualização, chamada de \textit{geodesic tracing}, obtém uma transformação de uma imagem original sobre a superfície.

A implementação feita nesse projeto é feita em três partes:
compilador, método numérico e interface gráfica.

O compilador fornece uma maneira do usuário definir as superfícies e outros objetos.
O usuário escreve um texto, seguindo algumas regras gramaticais, que então é processado.
A teoria de compiladores é essencial para essa etapa,
principalmente a análise léxica e a análise sintática \cite{Dragon:1}.
O compilador está descrito no capítulo \ref{compiler}.

O método numérico se refere à simulação dos raios de luz na superfície.
Um raio de luz é determinado pela posição e direção inicial, que são as condições iniciais.
Um sistema de equações diferenciais ordinárias(equação geodésica \cite{GeomDiff:1})
determina a curva que a luz traça.
Uma solução aproximada da equação é calculada pelo método de Runge-Kutta de ordem 4 \cite{Anal:1}.
O método está descrito no capítulo \ref{geomdiff}.

A interface gráfica é simples e é construída usando \textit{ImGui} \cite{ImGui},
uma ferramenta de interface gráfica fácil de usar.
A linguagem de programação escolhida para a implementação desse projeto é \textit{C++},
e para desenhar a interface e os objetos, \textit{OpenGL} é usado.