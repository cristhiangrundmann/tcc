% -----------------------------------
% -----------------------------------
% abnTeX2: Normas ABNT NBR 14724:2011 + sugestões FGV/EMAp. 

% Autor: Lauro César Araujo
% Adaptações EMAp: Lucas Machado Moschen 
% Copyright 2012-2018 by abnTeX2 group at http://www.abntex.net.br/ 

%% This work may be distributed and/or modified under the
%% conditions of the LaTeX Project Public License, either version 1.3
%% of this license or (at your option) any later version.
%% The latest version of this license is in
%%   http://www.latex-project.org/lppl.txt
%% and version 1.3 or later is part of all distributions of LaTeX
%% version 2005/12/01 or later.
% ----------------------------------
% ----------------------------------
\documentclass[
	% -- opções da classe memoir --
	12pt,				% tamanho da fonte
	%openright,			% capítulos começam em página ímpar (insere página vazia caso preciso)
	oneside,			% para impressão em recto e verso. Oposto a oneside
	a4paper,			% tamanho do papel. 
	% -- opções da classe abntex2 --
	%chapter=TITLE,		% títulos de capítulos convertidos em letras maiúsculas
	%section=TITLE,		% títulos de seções convertidos em letras maiúsculas
	%subsection=TITLE,	% títulos de subseções convertidos em letras maiúsculas
	%subsubsection=TITLE,% títulos de subsubseções convertidos em letras maiúsculas
	% -- opções do pacote babel --
	english,			% idioma para inglês
	brazil				% idioma para português
	]{abntex2}

%------------------------------------------------
%-------------- Pacotes necessários -------------
%------------------------------------------------

% Escrita 
\usepackage[T1]{fontenc}
\usepackage[utf8]{inputenc}
\usepackage{lmodern}
\usepackage{microtype} % para melhorias de justificação
\usepackage{indentfirst}

\renewcommand{\ABNTEXchapterfont}{\fontfamily{ptm}\fontseries{b}\selectfont}

% Gráficos 
\usepackage{color}
\usepackage{caption}
\usepackage{subcaption}
\usepackage{graphicx}
\graphicspath{{files/}}

% Matemáticos 
\usepackage{amsthm, amssymb, amsmath, mathtools}

% Outros 
%\usepackage{float}
\usepackage{tabularx}
\usepackage{listings}
\usepackage{tocloft}
\usepackage{newfloat}

\definecolor{programBack}{RGB}{43,43,43}
\definecolor{CC0}{RGB}{200, 200, 200}
\definecolor{CC1}{RGB}{30, 30, 30}
\definecolor{CC2}{RGB}{78, 201, 176}
\definecolor{CC3}{RGB}{86, 156, 214}
\definecolor{CC4}{RGB}{156, 220, 254}
\definecolor{CC5}{RGB}{78, 201, 176}
\definecolor{CC6}{RGB}{181, 206, 168}
\definecolor{CC7}{RGB}{197, 134, 192}
\definecolor{CC8}{RGB}{96, 161, 104}
\definecolor{CC9}{RGB}{200, 50, 5}

%\renewcommand{\lstlistlistingname}{Lista de códigos}
\renewcommand{\lstlistingname}{Código}
\lstset{basicstyle=\ttfamily\footnotesize,breaklines=true, numbers=left,
inputencoding=utf8, extendedchars=true, frame=single,
escapeinside={$}{$}, captionpos=b,
literate=%
        {é}{{\'{e}}}1
        {è}{{\`{e}}}1
        {ê}{{\^{e}}}1
        {ë}{{\¨{e}}}1
        {É}{{\'{E}}}1
        {Ê}{{\^{E}}}1
        {û}{{\^{u}}}1
        {ù}{{\`{u}}}1
        {ú}{{\'{u}}}1
        {â}{{\^{a}}}1
        {à}{{\`{a}}}1
        {á}{{\'{a}}}1
        {ã}{{\~{a}}}1
        {Á}{{\'{A}}}1
        {Â}{{\^{A}}}1
        {Ã}{{\~{A}}}1
        {ç}{{\c{c}}}1
        {Ç}{{\c{C}}}1
        {õ}{{\~{o}}}1
        {ó}{{\'{o}}}1
        {ô}{{\^{o}}}1
        {Õ}{{\~{O}}}1
        {Ó}{{\'{O}}}1
        {Ô}{{\^{O}}}1
        {î}{{\^{i}}}1
        {Î}{{\^{I}}}1
        {í}{{\'{i}}}1
        {Í}{{\~{Í}}}1
}

\AtBeginDocument{% the counter is defined later
  \counterwithout{lstlisting}{chapter}%
}

%remake table of contents for code listings
\newcommand{\listofcodesname}{Lista de códigos}
\newcommand{\codename}{Código}
\newlistof{codestoc}{locd}{\listofcodesname}
%\setcounter{locddepth}{1}
\newlistentry{codestocchapter}{locd}{0}
\renewcommand{\cftcodestocchaptername}{\codename\space}
\renewcommand*{\cftcodestocchapteraftersnum}{\hfill--\hfill}
\counterwithout{codestocchapter}{chapter}

\newcommand{\code}[3]{%
    \refstepcounter{codestocchapter}%
    \addcontentsline{locd}{codestocchapter}{{\protect\numberline{\thecodestocchapter}#2}}%
    \makeatletter%
    \lstinputlisting[caption=#2, label=#3]{#1}%
    \makeatother%
}

% Citações 
%\usepackage[brazilian,hyperpageref]{backref}
%\usepackage[alf]{abntex2cite}	% Citações padrão ABNT
\usepackage[style=abnt]{biblatex}
\addbibresource{biblio.bib}  

% \renewcommand{\backrefpagesname}{Citado na(s) página(s):~}
% % Texto padrão antes do número das páginas
% \renewcommand{\backref}{}
% % Define os textos da citação
% \renewcommand*{\backrefalt}[4]{
% 	\ifcase #1 %
% 		Nenhuma citação no texto.%
% 	\or
% 		Citado na página #2.%
% 	\else
% 		Citado #1 vezes nas páginas #2.%
% 	\fi}%
% ---

\input{files/capa_folha_rosto.tex}

\titulo{\textit{Geodesic Tracing}}
\subtitulo{Um sistema de especificação e visualização de curvas e superfícies através de geodésicas}
\autor{Cristhian Grundmann}
\orientador{Asla Medeiros e Sá}
\local{Rio de Janeiro}
\data{2022}
\instituicao{%
  Fundação Getulio Vargas \\
  \par
  Escola de Matemática Aplicada
}
\tipotrabalho{Trabalho de Conclusão de Curso}

\preambulo{Trabalho de conclusão de curso apresentada para a Escola de
Matemática Aplicada (FGV/EMAp) como requisito para o grau de bacharel em
Matemática Aplicada. \\ \\ Área de estudo: curvas e superfícies.}


%---------------------------------------------
%-------------------- PDF --------------------
%---------------------------------------------

% alterando o aspecto da cor azul
\definecolor{blue}{RGB}{41,5,195}

% informações do PDF
\makeatletter
\hypersetup{
     	%pagebackref=true,
		pdftitle={\@title}, 
		pdfauthor={\@author},
    	pdfsubject={\imprimirpreambulo},
	    pdfcreator={pdflatex},
		pdfkeywords={tcc}, 
		colorlinks=true,       		% false: boxed links; true: colored links
    	linkcolor=blue,          	% color of internal links
    	citecolor=blue,        		% color of links to bibliography
    	filecolor=magenta,      		% color of file links
		urlcolor=blue,
		bookmarksdepth=4
}
\makeatother

% Posiciona figuras e tabelas no topo da página quando adicionadas sozinhas
% em um página em branco. Ver https://github.com/abntex/abntex2/issues/170
\makeatletter
\setlength{\@fptop}{5pt} % Set distance from top of page to first float
\makeatother

%-----------------------------------------------------
%--------------------- Margens -----------------------
%-----------------------------------------------------

\setlrmarginsandblock{3cm}{2cm}{*}
\setulmarginsandblock{3cm}{2cm}{*}
\checkandfixthelayout

%-----------------------------------------------------
%------ Espaçamentos entre linhas e parágrafos -------
%-----------------------------------------------------

% O tamanho do parágrafo é dado por:
\setlength{\parindent}{1.3cm}

% Controle do espaçamento entre um parágrafo e outro:
\setlength{\parskip}{0.2cm}  % tente também \onelineskip

% compila o índice
\makeindex

%------------------------------------------------------
%----------- Personal Definitions ---------------------
%------------------------------------------------------

%\input{files/definitions.tex}

%-------------------------------------------------
%----------------- Document ----------------------
%-------------------------------------------------

\begin{document}

\newcounter{num}
% if num != 1, do not print the pre textual 
\setcounter{num}{1}

\selectlanguage{brazil}
\frenchspacing 

%----------------------------------------------
%--------------- Pré-textuais -----------------
%----------------------------------------------
\pretextual

\imprimircapa

\ifnum\value{num}=1
{\imprimirfolhaderosto*

%\input{files/ficha_catalografica.tex}
%\input{files/folha_aprovacao.tex}
%\input{files/dedicatoria_agradecimentos.tex}

\setlength{\absparsep}{18pt} 
\begin{resumo}[Resumo]
Curvas e superfícies costumam ser visualizados em um espaço ambiente 2D ou 3D.
Esse projeto implementa essa visualização em 3D, e para superfícies,
implementa também o \textit{Geodesic Tracing}: uma visualização intrínseca à
superfície, baseada em curvas geodésicas. Além de curvas e superfícies,
o projeto permite visualizar pontos e vetores. Outros objetos auxiliares podem ser
definidos, como parâmetros(controles deslizantes), funções e grades para
instanciar objetos múltiplas vezes.

Para a especificação dos objetos, uma linguagem textual foi estabelecida,
acompanhada de um compilador capaz de tranformar o texto em estruturas de dados
úteis para a renderização.

Para a interface gráfica, \textit{OpenGL} é usado para a renderização,
e \textit{Dear ImGUI} é usado para construir os controles e janelas.

...

 Palavras-chave: visualização. curvas. superfícies. compilador.
\end{resumo}

\begin{resumo}[Abstract]
 \begin{otherlanguage*}{english}
  (ENGLISH)
 \end{otherlanguage*}

 Keywords: ...
\end{resumo}

\pdfbookmark[0]{\listfigurename}{lof}
\listoffigures*
\cleardoublepage

\pdfbookmark[0]{\listtablename}{lot}
\listoftables*
\cleardoublepage

\pdfbookmark[0]{\listofcodesname}{locd}
\codestoc*
\cleardoublepage

}\fi

\pdfbookmark[0]{\contentsname}{toc}
\tableofcontents*
\cleardoublepage

% ----------------------------------------------------------
% ELEMENTOS TEXTUAIS
% ----------------------------------------------------------
\textual

\chapter{Introdução}
Existem diversos ambientes computacionais
bem estabelecidos para a funcionalidade de desenho de curvas e
superfícies a partir de descrições matemáticas, sejam paramétricas
ou implícitas. Alguns exemplos são o software \textit{Mathematica} \cite{mathematica},
\textit{Matlab} \cite{matlab}, seus similares em software livre(\textit{Octave}, \textit{Scilab},
linguagem \textit{Julia}), e softwares de Geometria Dinâmica, como o \textit{GeoGebra},
cuja descrição do objeto gráfico pode ser feito através da interface gráfica.

Tais softwares são compostos de três componentes cuja distinção nem sempre
é transparente ao usuário, são elas: especificação de um objeto gráfico,
renderização desse objeto, e sua interação com o usuário.
Nesse projeto, curvas, superfícies, pontos ou vetores são especificados
e então renderizados.

Tipicamente, a estratégia de renderização
de superfícies costuma assumir um ponto de vista no espaço ambiente 3D
como sendo o ponto de vista do observador.
Uma forma muito comum de renderização é a discretização da curva ou superfície,
formando segmentos no caso de uma curva, ou triângulos para superfícies.

O objetivo desse projeto é o desenvolvimento de um sistema de visualização
de curvas e superfícies. Para tanto, esse projeto implementa um sistema que inclui
a possibilidade de visualização de superfícies que não depende de um espaço ambiente.
A visualização simula um observador posicionado sobre a superfície, que percorre o mundo
restrito à essa dimensão.
Para isso, simula-se raios de luz partindo da posição do observador, e os pontos iluminados são observados.
Os raios de luz devem seguir caminhos em `linha reta', que minimizam distância.
Para uma superfície qualquer, esses caminhos são chamados de geodésicos,
estudados na geometria diferencial, e descritos no Capítulo \ref{geomdiff}.
A visualização, chamada de \textit{Geodesic Tracing}, renderiza a imagem sobre a superfície,
e suas curvaturas podem ser notadas. Ao se mover, a imagem observada pode se distorcer,
dependendo da curvatura.

A implementação desse projeto é feita em três partes:
compilador, método numérico e interface gráfica.

O compilador fornece uma maneira do usuário definir as superfícies e outros objetos.
O usuário escreve um texto, que então é processado.
O texto deve seguir uma gramática livre-de-contexto,
cuja in-ambiguidade foi provada.
A teoria de compiladores é essencial para essa etapa,
principalmente a análise léxica e a análise sintática \cite{Dragon:1}.
O compilador está descrito no Apêndice \ref{comp}.
A linguagem de especificação dos objetos gráficos, com exemplos de programas, está descrita no Capítulo \ref{lang}.

O método numérico se refere à simulação dos raios de luz na superfície.
Um raio de luz é determinado pela posição e direção inicial, que são as condições iniciais.
Um sistema de equações diferenciais ordinárias(equação geodésica \cite{GeomDiff:1})
determina a curva que a luz traça.
Uma solução aproximada da equação é calculada pelo método de Runge-Kutta de ordem 4 \cite{Anal:1}.
O método está descrito no Capítulo \ref{numeric}.

A interface gráfica é simples e é construída usando \textit{Dear ImGUI} \cite{ImGui},
uma ferramenta de interface gráfica fácil de usar.
A linguagem de programação escolhida para a implementação desse projeto é \textit{C++},
e para desenhar a interface e os objetos, \textit{OpenGL} é usado.
A interface está descrita no Capítulo \ref{interface}.

O objetivo primário desse projeto é a visualização de curvas, superfícies, e o \textit{Geodesic Tracing}.
Porém, a estética dos gráficos e da linguagem descritiva, performance
e robustez do sistema também são levados em consideração.

A Figura \ref{img:preview} demonstra a interface gráfica.

\begin{figure}[!ht]
    \centering
    \includegraphics[width=\linewidth, frame]{preview.png}
    \caption{Exemplo da visualização}
    \label{img:preview}
\end{figure}

\newpage

A Figura \ref{img:chart} representa os estágios da aplicação.

\begin{figure}[!ht]
    \centering
    \includegraphics[width=0.6\linewidth]{chart.png}
    \caption{Estágios da aplicação}
    \label{img:chart}
\end{figure}

A entrada se refere ao texto que o usuário escreve para definir os objetos matemáticos.
O texto é digitado na própria interface gráfica, numa caixa de texto multilinha
e com \textit{Syntax Highlighting}: cores associadas às palavras e símbolos.

A compilação se refere ao processamento da entrada.
Eventualmente, o usuário pode cometer erros sintáticos ou semânticos.
Esses erros são detectados no compilador, e uma mensagem de erro é emitida.
Se o texto for válido, várias estruturas de dados são geradas,
com todas as informações relevantes sobre os objetos. Além disso,
\textit{Shaders} são gerados e compilados.
Cada \textit{shader} faz a renderização de um objeto.
Para superfícies, um outro \textit{shader} é necessário para fazer o \textit{Geodesic Tracing}.

A saída se refere ao produto final da compilação.
Todos os objetos são desenhados na interface gráfica.
Além disso, o usuário pode controlar vários aspectos da visualização,
como a câmera, e controles deslizantes para os parâmetros.

Todos os arquivos necessários para a compilação dessa aplicação se encontram
no repositório público \cite{TCC}. Todos os códigos são abertos e livres.

% ----------------------------------------------------------
% Finaliza a parte no bookmark do PDF
% para que se inicie o bookmark na raiz
% e adiciona espaço de parte no Sumário
% ----------------------------------------------------------
\phantompart

\chapter{Curvas e Superfícies}
\label{geomdiff}
O objetivo primário do projeto é visualizar curvas e superfícies.
As curvas são visualizadas apenas em espaço 3D.
As superfícies são visualizadas em espaço 3D e em \textit{geodesic tracing}.

\section{Curvas}
Há duas principais maneiras de se definir uma curva na geometria analítica:
por parametrização e por equação. Esse trabalho apenas consideras curvas paramétricas.

Uma curva pode ser parametrizada por um número real.
Formalmente, uma parametrização é uma função $\gamma : I \rightarrow \mathbb{R}^n$, onde $I$ é um intervalo real.
Nesse trabalho, o intervalo é fechado, e $n=3$.

As curvas são desenhadas através de vários segmentos.
Dada uma partição de $I$ de $k$ pontos,
pode-se aproximar a curva pelos segmentos de extremidade $\gamma(t_{i})$ e $\gamma(t_{i-1})$ para $i<k$,
onde $t_i$ é o i-ésimo ponto da partição. Nesse trabalho, a partição depende apenas de $k$ e é uniforme.

O vetor tangente pode ser calculado com $\gamma'(t)$.

\section{Superfícies}
Assim como as curvas, apenas superfícies parametrizadas serão consideradas nesse trabalho:
$\sigma :  I_1 \times I_2 \rightarrow \mathbb{R}^3$, onde $I_1$ e $I_2$ são intervalos fechados reais.
Além disso, apenas superfícies regulares serão consideradas: a parametrização é suave e
os vetores tangentes são linearmente independentes.

As superfícies são desenhadas através de vários triângulos,
a partir de partições dos intervalos $I_1$ e $I_2$.
Juntas, as partições formam uma grade de retângulos, e cada retângulo pode ser dividido em 2 triângulos.
Esses são os triângulos desenhados.

Os vetores tangentes nas direções coordenadas são as derivadas parciais $\sigma_u(u,v)$ e $\sigma_v(u,v)$, onde os parâmetros são $u$ e $v$.

\subsection{Primeira forma fundamental}
Numa superfície parametrizada regular,
a primeira forma fundamental no ponto paramétrico $(u,v)$ é definida como
\[
    \left[
        \begin{array}{cc}
            \sigma_u \cdot \sigma_u & \sigma_u \cdot \sigma_v \\
            \sigma_v \cdot \sigma_u & \sigma_v \cdot \sigma_v
        \end{array}
    \right]
    = 
    \left[
        \begin{array}{cc}
            E & F \\
            F & G
        \end{array}
    \right]
\]
onde as funções são todas aplicadas no ponto $(u,v)$.

Os vetores $\sigma_u$ e $\sigma_v$ formam uma base do espaço tangente.
O produto escalar de dois vetores tangentes 
$x = x_1 \sigma_u + x_2 \sigma_v$ e $y = y_1 \sigma_u + y_2 \sigma_v$ pode ser calculado da seguinte forma:
\[x\cdot y = (x_1 \sigma_u + x_2 \sigma_v) \cdot (y_1 \sigma_u + y_2 \sigma_v)\]
\[x\cdot y = x_1 y_1 E + x_1 y_2 F + x_2 y_1 F + x_2 y_2 G\]

O produto depende apenas dos coeficientes e da primeira forma fundamental.
Isso significa que distâncias e ângulos podem ser calculados
sem se referir ao espaço ambiente da parametrização, ou seja, de forma intrínseca.

\subsection{Rotação}
Para a aplicação, é necessário rotacionar vetores no espaço tangente.
É possível rotacionar um vetor apenas usando seus componentes e
a primeira forma fundamental.

Seja $w = u\sigma_u+v\sigma_v$ o vetor a ser rotacionado e $\theta$ o ângulo da rotação.
A base $\{\sigma_u, \sigma_v\}$ do espaço tangente pode ser ortogonalizada.
Com uma base ortogonal, a rotação é feita com a matriz de rotação.

A nova base pode ser escrita como:
\begin{align*}
    \sigma'_u &= \frac{\sigma_u}{E}\\
    \sigma'_v &= \frac{\sigma_v-\frac{F\sigma_u}{E}}{R}\\
    R &= \sqrt{EG-F^2}
\end{align*}

Note que os vetores são ortogonais e de mesmo comprimento.
Apesar do comprimento não ser 1, a matriz de rotação funciona corretamente.

Para achar os componentes de $w$ na nova base, basta observar:
$w = u'\sigma'_u+v'\sigma'_v = u'\frac{\sigma_u}{E} + v'\frac{\sigma_v}{R}-v'\frac{\sigma_uF}{RE}$

Então
\begin{align*}
    u' &= uE + vF\\
    v' &= vR
\end{align*}

O vetor rotacionado é
\[w' = \left(u'\text{cos}\theta-v'\text{sin}\theta\right)\sigma'_u + \left(u'\text{sin}\theta+v'\text{cos}\theta\right)\sigma'_v\]
\[w' = \left(u\text{cos}\theta-\frac{uF+vG}{R}\text{sin}\theta\right)\sigma_u+\left(\frac{uE+vF}{R}\text{sin}\theta+v\text{cos}\theta\right)\sigma_v\]

Como esperado, esse vetor só depende da primeira forma fundamental,
dos componentes originais e do ângulo de rotação.

\subsection{Transporte Paralelo}
Seja $\gamma$ uma curva sobre a superfície e $v$ um campo vetorial tangente sobre a curva $\gamma$.
Diz-se que $v$ é paralelo ao longo de $\gamma$ quando $v'$ é ortogonal à superfície
para qualquer ponto de $\gamma$.
Nesse caso, um ser sobre a superfície não observaria mudanças em $v$ ao traçar a curva $\gamma$.
A variação de $v$ se dá ortogonalmente à superfície, logo não seria percebida pelo ser.
Diz-se que o vetor $v$ está sendo transportado paralelamente ao longo de $\gamma$.

Sejam $v$ e $w$ vetores paralelos ao longo de $\gamma$(transportados paralelamente).
Então o produto escalar $v \cdot w$ é constante, pois $(v \cdot w)' = v' \cdot w + v \cdot w' = 0$.
Como o produto escalar pode definir as noções de comprimento e ângulo, vetores transportados
paralelamente à uma curva mantém seus comprimentos e ângulos entre si.

\subsection{Equação geodésica}
Uma curva regular $\gamma$ sobre a superfície é dita geodésica quando
$\gamma'$ é paralelo ao longo de $\gamma$.
Nesse caso, um ser perceberia $\gamma'$(velocidade) como constante, e a curva $\gamma$ pode
ser considerada como reta nesse espaço.

Uma curva $\gamma$ é uma curva geodésica quando satisfaz o sistema \ref{geoeq}.
\begin{equation}
\label{geoeq}
    \left\{\begin{array}{cc}
        \dfrac{1}{2}\left(E_u(u')^2 + 2F_uu'v' + G_u(v')^2\right) & = \dfrac{d}{dt}(Eu' + Fv') \vspace{6pt}\\
        \dfrac{1}{2}\left(E_v(u')^2 + 2F_vu'v' + G_v(v')^2\right) & = \dfrac{d}{dt}(Fu' + Gv')
    \end{array}\right.
\end{equation}

Esse sistema pode ser reescrito como o sistema \ref{geoeq2}.

\begin{equation}
\label{geoeq2}
        2
        \begin{pmatrix}
            E & F \\
            F & G
        \end{pmatrix}
        \begin{pmatrix}
            u'' \\
            v''
        \end{pmatrix}
        =
        \begin{pmatrix}
            v'^2(G_u-2F_v)-u'^2E_u-2u'v'E_v \\
            u'^2(E_v-2F_u)-v'^2G_v-2u'v'G_u
        \end{pmatrix}
\end{equation}

Como a superfície é regular, $u''$ e $v''$ são únicos, pois a primeira forma
fundamental é invertível.

Note que a aceleração $(u'', v'')$ pode ser obtida em função da posição $(u,v)$
e velocidade $(u', v')$: $(u'', v'') = g(u, v, u', v')$.

Esse é um sistema de equações diferenciais ordinárias de primeira ordem.
Sua solução depende de uma posição e uma velocidade inicial.
Na maioria das superfícies interessantes, esse sistema é muito difícil, ou até impossível,
de resolver analiticamente.

\subsection{Geodesic Tracing}
A visualização do Geodesic Tracing requer uma imagem sobre a superfície.
Para isso, uma cor é associada a cada ponto $(u,v)$ do domínio da parametrização.

O geodesic tracing gera uma imagem a partir de duas informações:
\begin{itemize}
    \item um ponto $(u,v)$ da superície, representando a posição da câmera
    \item dois vetores tangentes $X$ e $Y$ no ponto $(u,v)$.
    Os vetores $X$ e $Y$ são de mesmo comprimento($z$), são ortogonais entre si e
    representam a orientação da câmera.
\end{itemize}

A imagem gerada é um quadrado $[-1, +1] \times [-1, +1]$.
O primeiro eixo corresponde ao vetor $X$, e o segundo ao $Y$.

Cada ponto $(x,y)$ é associado a um ponto $(u_1, v_1)$ da superfície,
e logo a um ponto da imagem original, obtendo-se uma cor.
O ponto $(u_1, v_1)$ é determinado traçando-se uma curva geodésica $\gamma$.
A posição inicial de $\gamma$ é $(u,v)$, e a velocidade inicial é $xX+yY$.
Finalmente, o ponto $(u_1, v_1)$ é definido como $\gamma(1)$.
A distância percorrida, ou comprimento de arco, é o próprio comprimento
da velocidade inicial: $z\sqrt{x^2+y^2}$.

A figura \ref{img:geotracing} ilustra o geodesic tracing.
Os vetores em rosa são os vetores $X$ e $Y$, ancorados na câmera, o ponto amarelo.
O vetor azul é uma combinação linear de $X$ e $Y$, e portanto está no espaço tangente.
O ponto da superfície observado pelo vetor azul é o final da curva vermelha, o ponto azul.
A curva é uma geodésica com velocidade initial igual ao vetor azul, partindo do ponto amarelo.

\begin{figure}[!ht]
    \includegraphics[width=\linewidth]{geotracing.png}
    \caption{Ilustração do geodesic tracing}
    \label{img:geotracing}
\end{figure}

\subsection{Interação}
Para compreender melhor o geodesic tracing,
é necessário alterar as condições e observar as alterações na imagem.

A interação mais simples é a rotação.
Os vetores $X$ e $Y$ são apenas rotacionados por um ângulo $\theta$.
A imagem resultante não se deforma, apenas é rotacionada pelo mesmo ângulo $\theta$.

Outra interação simples é o \textit{zoom}.
Os vetores $X$ e $Y$ são multiplicados por um fator $\alpha > 0$.
Para $\alpha > 1$, a imagem é ampliada, pois as curvas geodésicas são maiores.
Para $\alpha < 1$, a imagem é contraída.

A interação mais interessante é o movimento.
Para se mover na direção $X$, uma geodésica $\gamma$ é traçada na direção $X$.
O novo centro da imagem passa a ser $\gamma(t)$, onde $t$ é o tamanho do passo.
O novo vetor $X$ é a velocidade final($\gamma'(t)$),
que foi transportada paralelamente ao longo
de $\gamma$. O comprimento de $X$ foi preservado.
O novo vetor $Y$ também é transportado paralelamente.
Como visto anteriormente, seu comprimento é preservado e seu ângulo com $X$ também.
Assim, o novo $Y$ é apenas a rotação de $X$ pelo ângulo de $90$ graus.
O comprimento real do passo é $zt$.
De forma similar, a câmera pode se movimentar ao longo de $Y$.

As curvaturas da superfície causam distorções na imagem observada.
Ao se mover, pode-se perceber curvatura pela forma com que a imagem se distorce.
Curvatura gaussiana positiva faz os ``objetos'' expandirem ao se afastar.
Curvatura negativa faz os ``objetos'' contraírem ao se afastar.
\chapter{Linguagem de especificação dos objetos gráficos}
\label{lang}

O usuário se comunica com a interface através de um texto, chamado de programa,
que contém os objetos de interesse. 
Por exemplo, em \textit{GeoGebra}, o círculo unitário pode ser
definido como
\texttt{c = Curve(cos(t), sin(t), t, 0, 2pi)}.
Na linguagem desse projeto, a definição seria
\texttt{curve c(t) = (cos(t), sin(t), 0), t : [0, 2pi];}.

O Código \ref{ex1} é um exemplo.
\lstset{backgroundcolor=\color{programBack}}\code{files/ex1.col.txt}{Código para a Figura \ref{img:ex1}}{ex1}

A linguagem permite comentários no estilo da linguagem Python, usando \texttt{\#}.

O programa declara os seguintes objetos:

\begin{table}[ht]
\caption{Objetos do Código \ref{ex1}}
\label{objtypes}
\begin{centering}
\begin{tabularx}{\textwidth}{||c|X||}
    \hline
    \texttt{r} e \texttt{o} & parâmetros que podem ser alterados na interface.
    Seus valores devem estar nos intervalos indicados. \\ 

    \hline
    \texttt{c} & uma curva parametrizada por \texttt{t}.
    O domínio da parametrização é o intervalo indicado.
    A curva depende do parâmetro \texttt{r}, que foi definido anteriormente. \\

    \hline
    \texttt{k} & uma grade de 8 pontos igualmente espaçados no intervalo indicado.
    Uma grade é tratada como uma constante, assim como um parâmetro.
    Se um objeto desenhável depende de uma grade, uma instância é desenhada para cada valor da grade.
    Um objeto pode depender de mais de uma grade. \\

    \hline
    \texttt{k2} & uma constante, e não pode ser alterada na interface como os parâmetros.
    Esse tipo de objeto pode ser usado para deixar o programa mais legível. \\
    
    \hline
    \texttt{p} & o ponto da curva \texttt{c} de parâmetro \texttt{t = k2}.
    Esse objeto depende indiretamente de \texttt{k}, então é instanciado 8 vezes. \\
    
    \hline
    \texttt{v} & o vetor tangente da curva \texttt{c} no ponto \texttt{p} 
    e desenhado a partir do mesmo ponto.
    O vetor também depende indiretamente de \texttt{k}, então é desenhado 8 vezes. \\
    \hline
\end{tabularx}
\end{centering}
\end{table}

A Figura \ref{img:ex1} demonstra os objetos declarados em perspectiva 3D.
\begin{figure}[!ht]
    \centering
    \includegraphics[width=0.6\linewidth, frame]{ex1.png}
    \caption{Renderização do Código \ref{ex1}}
    \label{img:ex1}
\end{figure}

Os objetos \texttt{f} e \texttt{s} estão comentados, então não são considerados.
Estão presentes apenas para o exemplo ter todos os tipos de objeto.

A linguagem de descrição de objetos não é trivial, nem sua sintaxe matemática,
que possui elementos inventados para esse projeto.
A seguir, uma breve lista de observações:

\begin{itemize}
\item
Os objetos desenháveis são pontos, vetores, curvas e superfícies.
Pontos e vetores podem ser usados em outros objetos, sendo tratados como tuplas.
Por exemplo, \texttt{v} usa o ponto \texttt{p}.
Curvas e superfícies podem ser usadas como funções, mas sem a restrição no domínio.
Por exemplo, \texttt{p} usa \texttt{c} como função.
Um objeto só pode se referir aos objetos definidos anteriormente.
Os parâmetros das curvas e superfícies podem ter qualquer nome disponível.

\item
Há duas constantes pré-definidas: \texttt{pi} e \texttt{e};
e diversas funções pré-definidas:
\texttt{sin}, \texttt{cos}, \texttt{tan},
\texttt{exp}, \texttt{log}, \texttt{sqrt} e \texttt{id}.
A função \texttt{id} é a identidade e é útil apenas no funcionamento interno do sistema.

\item
Parâmetros e grades podem ser multidimensionais:
\texttt{param T : [0, 1], [0, 1];}. Assim, o objeto \texttt{T} é uma tupla,
e seus elementos podem ser obtidos com \texttt{T\_1} e \texttt{T\_2}.

\item As grades das curvas e superfícies são por padrão 100 e 100x100, respectivamente.
É possível alterar esse valor informando um intervalo do
tipo grade: \texttt{[0, 2pi, 250]}.

\item
Há 4 operadores unários. Os operadores \texttt{+} e \texttt{-} são usuais:
podem representar as operações binárias simples, como \texttt{1+3}, ou podem
representar sinais como operações unárias, por exemplo \texttt{-x}.
A operação \texttt{*x} representa \texttt{xx}, e \texttt{/x} é igual a \texttt{1/x}.
Para números reais, multiplicação com \texttt{*} e por justaposição são equivalentes.
Porém, para tuplas, \texttt{a*b} representa o produto vetorial
e \texttt{ab} representa o produto escalar.
Assim, \texttt{*x} calcula o quadrado do módulo do vetor \texttt{x}.
Uma função que normaliza vetores pode ser
definida assim: \texttt{function N(x) = x/sqrt*x;}.

\item
Numa aplicação de função de uma variável, o argumento não precisa de parênteses:
\texttt{sin x}.
O argumento pode ter operadores unários e até expoentes:
\texttt{sin -x\textasciicircum2 = sin(-x\textasciicircum2)}.
Deve-se tomar cuidado com expoentes:
\texttt{sin(x)\textasciicircum y = sin(x\textasciicircum y)}.
Para a exponenciação de uma aplicação,
deve-se usar a sintaxe: \texttt{sin\textasciicircum2 x}.

\item
Não é sempre necessário uma separação entre identificadores.
Por exemplo, considere \texttt{sinx}.
Caso haja um termo chamado \texttt{sinx} definido, esse seria o
identificador reconhecido.
Caso contrário, \texttt{sin x} será reconhecido,
mesmo que \texttt{sinx} seja definido posteriormente
(\texttt{sinx} seria reconhecido apenas depois de sua definição).
Em geral, o maior identificador definido será reconhecido.
\end{itemize}

A especificação completa da gramática está descrita no Código \ref{grammar}
do Apêndice \ref{compiler}.

\newpage

O Código \ref{ex2} é outro exemplo.
\lstset{backgroundcolor=\color{programBack}}\code{files/ex2.col.txt}{Código para as Figuras \ref{img:ex2} e \ref{img:ex2gt}}{ex2}

\begin{figure}[!ht]
\centering
\subfloat[Perspectiva em 3D]{\includegraphics[width=0.42\linewidth, frame]{ex2.png}%
\label{img:ex2}}
\hfil
\subfloat[\textit{Geodesic Tracing}]{\includegraphics[width=0.42\linewidth, frame]{ex2gt.png}%
\label{img:ex2gt}}
\caption[Renderizações do Código \ref{ex2}]
{A Figura \ref{img:ex2} à esquerda é a renderização em perspectiva 3D.
A Figura \ref{img:ex2gt} à direita é a renderização em \textit{Geodesic Tracing}}
\label{img:ex2Both}
\end{figure}

\begin{figure}[!ht]
    \centering
    \includegraphics[width=0.6\linewidth, frame]{map.png}
    \caption[Textura utilizada na Figura \ref{img:ex2Both}]{Textura utilizada na Figura \ref{img:ex2Both} \\ Versão modificada de \url{https://commons.wikimedia.org/wiki/File:Equirectangular\_projection\_SW.jpg}}
    \label{img:original}
\end{figure}

Para o geodesic tracing na Figura \ref{img:ex2gt} é possível observar anomalias para
os `raios' que chegam muito perto dos polos. Essa anomalia é uma instabilidade numérica
causada pela parametrização da superfície.
\chapter{Método numérico}
\label{numeric}

A equação \ref{geoeq2} caracteriza as curvas geodésicas que devem ser 
computadas para a visualização do Geodesic Tracing.
Nem sempre é possível resolver a equação de forma analítica, então uma aproximação
deve ser computada no lugar.

O método numérico escolhido é o método de Runge-Kutta de ordem 4 \cite{Anal:1}.
Seja $y' = f(t, y)$ uma equação diferencial para $y(t)$ com valores iniciais $t_0$ e $y_0$.
O método consiste em aproximar $y_1 = y(t_0+h)$, para um passo $h>0$.
A aproximação é feita pelas equações \ref{runge}.

\begin{equation}
\label{runge}
\begin{split}
k_1 & = hf(t_0, y_0) \\
k_2 & = hf(t_0 + h/2, y_0 + k_1/2) \\
k_3 & = hf(t_0 + h/2, y_0 + k_2/2) \\
k_4 & = hf(t_0 + h, y_0 + k_3) \\
y_1 & = y_0 + (k_1 + 2k_2 + 2k_3 + k_4)/6 \\
t_1 & = t_0 + h
\end{split}
\end{equation}

As equações devem ser adaptadas para o sistema \ref{geoeq2}, pois é de 
segunda ordem e possui mais de uma equação.
O sistema pode ser escrito vetorialmente:
\[y' = (u, v, u', v')' = f(u, v, u', v') = f(y)\]

Ou ainda 
\[y' = (\text{pos}, \text{vel})' = f(\text{pos}, \text{vel}) = f(y)\]

A função $f$ computa $(u', v', u'', v'') = (\text{vel}, \text{acc})$, onde $\text{vel}$ é
igual ao argumento $\text{vel}$ de $f$. Os valores $u''$ e $v''$ são calculados conforme a equação
\ref{geoeq2}, usando $\text{pos}$ e $\text{vel}$.
Note que a função $f$ não depende do parâmetro $t$ da curva.

O sistema pode ser resolvido numericamente por Runge-Kutta, ignorando a variável $t$,
e fazendo $y$ como $(\text{pos}, \text{vel})$.

\begin{equation}
\label{method}
\begin{split}
    p_1 & = h\text{vel}_0 \\
    v_1 & = hg(\text{pos}_0, \text{vel}_0) \\
    p_2 & = h(\text{vel}_0+v_1/2) \\
    v_2 & = hg(\text{pos}_0+p_1/2, \text{vel}_0+v_1/2) \\
    p_3 & = h(\text{vel}_0+v_2/2) \\
    v_3 & = hg(\text{pos}_0+p_2/2, \text{vel}_0+v_2/2) \\
    p_4 & = h(\text{vel}_0+v_3) \\
    v_4 & = hg(\text{pos}_0+p_3, \text{vel}_0+v_3) \\
    \text{pos}_1 & = (p_1+2p_2+2p_3+p_4)/6 \\
    \text{vel}_1 & = (v_1+2v_2+2v_3+v_4)/6
\end{split}
\end{equation}
\chapter{Interface gráfica}
\label{interface}

A interface gráfica é responsável pela interação com o usuário.
Nela, o usuário escreve o programa, visualiza e interage com os objetos.

Uma janela, ``Program'', contém um caixa de texto multilinha, onde
o usuário deve escrever o programa.
Um botão, ``Compile'', compila o texto escrito.
A resposta do compilador será ``Status: OK'' para um programa compilado 
corretamente, ou ``Status: ...'' com uma mensagem de erro, indicando a
linha e a coluna correspondentes.
Dois controles deslizantes definem o tamanho, em pixels, da janela
de visualização 3D(``Frame size'') e do Geodesic Tracing(``Geo Size'').
Um botão, ``Open Texture'', serve para carregar uma imagem local,
com o nome informado à caixa de texto justaposta.

A figura \ref{img:program} ilustra a janela ``Program''.

\begin{figure}[!ht]
    \includegraphics[width=\linewidth]{program.png}
    \caption{Janela ``Program''}
    \label{img:program}
\end{figure}

Quando o programa é compilado corretamente, duas janelas extras são exibidas.

A janela ``Settings'' controla algumas propriedades dos objetos.
Para os parâmetros, um controle deslizante é criado, com a opção ``Animate''.
Quando ativado, o parâmetro é controlado pelo tempo,
crescendo uma unidade por segundo.
Quando o limite superior do parâmetro é atingido,
o valor volta para o limite inferior.

Para cada objeto desenhável é possível escolher uma cor.

Para uma superfície, os componentes RGB da textura
são multiplicadas pelos componentes RGB da cor escolhida.
A cor branca deixa a textura inalterada,
e preto deixa a textura completamente preta.
Além disso, é possível escolher uma textura previamente carregada para
uma superfície. A textura padrão é a textura local de nome ``default.png''.

A velocidade da câmera é controlada por ``Camera speed''.
Esse valor afeta as câmeras das janelas ``3D view'' e de ``Geodesic Tracing''.

A figura \ref{img:settings} ilustra a janela ``Settings''.

\begin{figure}[!ht]
    \includegraphics[width=\linewidth]{settings.png}
    \caption{Janela ``Settings''}
    \label{img:settings}
\end{figure}

A janela ``3D View'' exibe os objetos desenháveis no espaço 3d.
A câmera pode ser controlada da seguinte forma:

\begin{table}[ht]
\caption{Controles da câmera 3D}
\label{camctrl}
\begin{centering}
\begin{tabularx}{\textwidth}{||c|X||}
    \hline
    \texttt{W} & move a câmera para frente \\
    \hline
    \texttt{S} & move a câmera para trás \\
    \hline
    \texttt{A} & move a câmera para a esquerda \\
    \hline
    \texttt{D} & move a câmera para a direita \\
    \hline
    \texttt{Q} & move a câmera para cima (absoluto) \\
    \hline
    \texttt{E} & move a câmera para baixo (absoluto) \\
    \hline
    \texttt{Click \& move} & gira a câmera conforme o movimento do mouse \\
    \hline
\end{tabularx}
\end{centering}
\end{table}

A renderização dos objetos utiliza um antiserrilhado, melhorando sua estética.
Além disso, o desenho das linhas considera uma grossura que leva
em consideração a perspectiva, assim como os pontos e vetores.

Para uma superfície ``X'', a janela ``Geodesic Tracing - X'' é exibida.
A janela exibe a visualização do geodesic tracing.

A câmera pode ser controlada da seguinte forma:

\begin{table}[ht]
\caption{Controles do geodesic tracing}
\label{gtctrl}
\begin{centering}
\begin{tabularx}{\textwidth}{||c|X||}
    \hline
    \texttt{W} & move a câmera para frente \\ 
    \hline
    \texttt{S} & move a câmera para trás \\
    \hline
    \texttt{A} & move a câmera para a esquerda \\
    \hline
    \texttt{D} & move a câmera para a direita \\
    \hline
    \texttt{Q} & gira a câmera no sentido anti-horário \\
    \hline
    \texttt{E} & gira a câmera no sentido horário \\
    \hline
    \texttt{Z} & zoom in \\
    \hline
    \texttt{X} & zoom out \\
    \hline
    \texttt{Click \& move} & move a câmera conforme o movimento do mouse \\
    \hline
\end{tabularx}
\end{centering}
\end{table}

Dependendo da expansão ou contração da imagem gerada no geodesic tracing,
o tamanho dos pixels pode ficar grandes ou pequenos demais. O primeiro caso
deixa os pixels individuais evidentes, e o segundo gera ruído, pois pixels distantes
de cores muito diferentes ``competem'' para serem exibidos.
A exibição da textura é feita com filtro de magnificação linear, fazendo as transições
de pixels grandes mais suave, resolvendo o primeiro problema.
O segundo problema foi resolvido com a técnica de \textit{mipmapping} linear.
Nessa técnica, a textura é copiada em resoluções progressivamente menores.
Por exemplo, um tabuleiro de xadrez possui casas brancas e pretas, e sua menor resolução
é apenas um píxel cinza. Desse modo, quando a textura está muito contraída, 
uma versão de resolução menor é usada, resolvendo o ruído.
Essas técnicas podem ser facilmente obtidas configurando o OpenGL,
como pode ser observado em \cite{LearnOpenGL}.

\chapter{Conclusão}

A interface gráfica, a linguagem de especificação e o compilador
funcionam sem irregularidades ou inconsistências dentro de um conjunto de testes\footnote{Os testes se encontram no arquivo \texttt{app/sample.txt} do repositório do projeto \cite{TCC}},
e possuem performance em tempo-real.
Além disso, a estética da interface e da linguagem foi levada em consideração:
a linguagem possui \textit{Syntax Highlighting}, e a interface é construída usando uma biblioteca
terceira. O desenho de objetos gráficos é feito com anti-serrilhado, melhorando a estética.
Para o \textit{Geodesic Tracing}, é possível observar os fenômenos de curvatura
correspondentes à teoria da geometria diferencial.

Uma das limitações desse projeto é a linguagem de especificação.
Curvas e superfícies devem ser definidas por equações paramétricas.
Muitas curvas e superfícies são melhores definidas implicitamente.
Outra limitação é o fato do programa ser estrito apenas em forma textual, sem poder utilizar
notações matemáticas como por exemplo, subscritos e expoentes, frações e raízes.

Um possível trabalho futuro é aprimorar a discretização das curvas e superfícies.
Assim, cada segmento ou triângulo se ajusta melhor à curva ou superfície.
Para o desenho das superfícies em 3D, truques de textura podem ser usados.
Com os vetores normais à superfície, é possível fazer um sombreado na superfície(sem projeção de sombras),
aumentando a compreensão e o realismo da forma da superfície.
Com isso, é possível reduzir a quantidade de segmentos ou triângulos da discretização
sem degradar o realismo.
O \textit{Geodesic Tracing} desconsidera o domínio da parametrização da superfície.
Isso significa que superfícies como a faixa de \textit{M\"obius} e a garrafa de \textit{Klein}
não são exibidas de forma totalmente correta, pois deveria ser possível inverter a orientação da câmera
ao dar um passeio paralelo.
Outra possibilidade seria a especificação da superfície em um Atlas,
resolvendo as anomalias observadas na Figura \ref{img:ex2gt}.

% -----------------------------------
% ELEMENTOS PÓS-TEXTUAIS
% -----------------------------------
\postextual
% ----------------------------------

%\bibliography{biblio}
\printbibliography

%\glossary

% ----------------------------------------------------------
% Apêndices
% ----------------------------------------------------------

% ---
% Inicia os apêndices
% ---
\begin{apendicesenv}
\partapendices
\chapter{Programas}

%exemplo
O usuário se comunica com a interface através de um texto, chamado de programa, que contém as objetos de interesse.
Por exemplo:
\begin{lstlisting}[caption=Exemplo de objetos,label=code1]
#circle and tangents
param r : [/2, 1];
param o : [0, 2pi];
curve c(t) = r(cost, sint, 0), t : [0, 2pi];
grid k : [0, 2pi, 8];
define k2 = k + o;
point p = ck2;
vector v = c'k2 @ p;
\end{lstlisting}

O programa começa com uma linha de comentário, estilo Python.
Os objetos \texttt{r} e \texttt{o} são parâmetros nos intervalos indicados e \texttt{c} é um círculo
parametrizado por \texttt{t} no intervalo indicado($/2 = 1/2$).
Os parâmetros podem ser alterados por controles deslizantes na interface.
O objeto \texttt{k} é uma grade de 8 pontos igualmente espaçados no intervalo indicado.
A definição \texttt{k2} serve apenas de conveniência, usada no ponto \texttt{p} e no vetor \texttt{v}.
Note que \texttt{p = ck2} ($p = c(k2)$) usa uma notação sem parênteses para a aplicação de funções.
Como o ponto \texttt{p} depende de \texttt{k}, 
um ponto é desenhado para cada valor de \texttt{k} da grade.
O mesmo ocorre para o vetor \texttt{v}. As grades são tratadas como constantes,
e servem para desenhar múltiplas instâncias dos objetos desenháveis.

%regras semânticas
Os objetos só podem se referir aos objetos declarados anteriormente.
Aplicações de funções devem ter a quantidade certa de argumentos.
Operações com tuplas devem ter quantidades consistentes de elementos.
Componentes devem ter índices corretos.
Intervalos não podem depender de parâmetros ou grades.


%gramática
O programa deve seguir uma gramática formal que determina as estruturas sintáticas permitidas.
A gramática especifica: o formato da declaração de cada tipo de objeto;
e as operações matemáticas e suas ordens de precedência.
A gramática é definida pelo código \ref{gram}.

\newpage
\begin{lstlisting}[caption=Gramática livre de contexto,label=gram]
PROG    = DECL PROG | ;

DECL    = "param"     id ":" INTS ";" ;
DECL    = "grid"      id ":" GRIDS ";" ;
DECL    = "define"    id "=" EXPR ";" ;
DECL    = "curve"     FDECL "," TINTS ";" ;
DECL    = "surface"   FDECL "," TINTS ";" ;
DECL    = "function"  FDECL ";" ;
DECL    = "point"     id "=" EXPR ";" ;
DECL    = "vector"    id "=" EXPR "@" EXPR ";" ;

FDECL   = id "(" IDS ")" "=" EXPR ;
IDS     = IDS "," id | id ;
INT     = [ EXPR "," EXPR ] ;
GRID    = [ EXPR "," EXPR "," EXPR ] ;
TINT    = id ":" INT ;
INTS    = INTS "," INT | INT ;
TINTS   = TINTS "," TINT | TINT ;
GRIDS   = GRIDS "," GRID | GRID ;

EXPR    = ADD;
ADD     = ADD "+" JUX | ADD "-" JUX | JUX ;
JUX     = JUX MULT2 | MULT ;
MULT    = MULT "*" UNARY | MULT "/" UNARY | UNARY ;
MULT2   = MULT2 "*" UNARY | MULT2 "/" UNARY | APP ;
UNARY   = "+" UNARY ;
UNARY   = "-" UNARY ;
UNARY   = "*" UNARY ;
UNARY   = "/" UNARY ;
UNARY   = APP ;
APP     = FUNC UNARY | POW ;
FUNC    = FUNC2 "^" UNARY | FUNC2 ;
FUNC2   = FUNC2 "_" var | FUNC2 "'" | func ;

POW     = COMP "^" UNARY | COMP ;
COMP    = COMP "_" num | FACT ;
FACT    = const | num | var | "(" TUPLE ")" | "[" TUPLE "]" | "{" TUPLE "}" ;
TUPLE   = ADD "," TUPLE | ADD ;
\end{lstlisting}

Os termos à esquerda de uma igualdade são os não-terminais, em maiúsculo, e \texttt{PROG} é o não-terminal inicial.
Os termos em minúsculo são terminais, e representam um token.
Os termos entre aspas representam literalmente o texto entre aspas.
Por exemplo, \texttt{func} representa o nome de uma função. Termos
O lado direito de uma igualdade especifica as possíveis formas sentenciais do 
não-terminal à esquerda, separadas por \texttt{|}.
Uma forma sentencial pode conter símbolos terminais e não-terminais.
Uma sentença, ou programa, é gerado a partir de \texttt{PROG}.
Iterativamente, se substitui um não-terminal por qualquer uma de suas formas sentenciais. 
Note que a primeira substituição é a de \texttt{PROG}.
A estrutura do programa é determinada por quais foram as substituições feitas.
Observe na linha 10 a especificação de um vetor. A palavra chave \texttt{vector} é necessária, seu nome é representado por \texttt{id} (identificador),
e os termos \texttt{EXPR} representam expressões matemáticas.

O não-terminal \texttt{EXPR} está definido na linha 21, e representa uma expressão matemática.
A linha 22 especifica os operadores do nível de precedência aditiva.

%%%....
Implementação de sintaxe que considera a estetica natural da escrita matemática que precisa ser traduzida para sintaxe de linguagem computacional. Isso é feito pelo parser.
A justificativa para essa abordagem é motivada pela experiencia de especificar desenhos de objetos gráficos em bibliotecas de uso corrente, tais como sagemath, manin etc.
Isso justifica a especificação de uma gramática livre de contexto. Explicitada a seguir.
Trabalhos futuros: grade variável
Sobre o texto que descreve a gramática
Fazer referencia à calculadora C++
Fazer referência ao site que valida a não ambiguidade da gramática
Por ter influenciado na sintaxe proposta (como?)
Tentar lembrar um exemplo de ambiguidade que foi resolvido para a versão atual da gramática.
Descrever o LL1 como certificado de não ambiguidade.
\end{apendicesenv}
% ---

% ----------------------------------------------------------
% Anexos
% ----------------------------------------------------------

% \begin{anexosenv}

% \partanexos

% \end{anexosenv}

%---------------------------------------------------------------------
% ÍNDICE REMISSIVO
%---------------------------------------------------------------------
\phantompart
\printindex

\end{document}