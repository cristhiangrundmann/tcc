\documentclass[10pt,a4paper,final]{article}
\usepackage[utf8]{inputenc}
\usepackage[T1]{fontenc}
\usepackage[main=portuguese, english]{babel}
\usepackage{amsmath}
\usepackage{amsfonts}
\usepackage{amssymb}
\usepackage{graphicx}
\usepackage{enumitem}
\usepackage{listings}
\usepackage{tcolorbox}

\author{Cristhian Grundmann}
\title{Visualização de curvas e superfícies a partir de geodésicas}

\begin{document}

\maketitle

%\tableofcontents


\section{Gramática de descrição de objetos}
A gramática permite a declaração de 8 tipos de objeto: parâmetro, curva, superfície, definição, função, grade, ponto e vetor.

Os objetos podem fazer referência apenas a objetos definidos anteriormente, porém apenas pontos e vetores podem se referir a grades, diretamente ou indiretamente.

A gramática descreve duas estruturas principais: as declarações dos objetos(\texttt{DECL}) e as expressões matemáticas(\texttt{ADD}). As declarações têm estrutura simples, já as expressões são mais complexas.

A gramática das expressões matemáticas é mais especial quando comparada às linguagens de programação gerais, pois o domínio desse projeto é muito mais limitado. As partes não usuais são:

\begin{itemize}
\item Multiplicação justaposta: \texttt{2 x y z}
\item Função sem parênteses: \texttt{sin -x}
\item Recíproco unário: \texttt{/x = 1/x}
\item Múltiplicação unária(sem efeito): \texttt{*x = x}
\item Potenciação e derivação em funções: \texttt{f\textasciicircum 2 x}, \texttt{sin' x}
\end{itemize}


\begin{tcolorbox}
\begin{small}
\lstinputlisting[title=Gramática completa]{grammar.txt}
\end{small}
\end{tcolorbox}



\end{document}